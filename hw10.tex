% HMC Math dept HW template example
% v0.04 by Eric J. Malm, 10 Mar 2005
\documentclass[12pt,letterpaper,boxed]{hmcpset}

% set 1-inch margins in the document
\usepackage[margin=1in]{geometry}

% include this if you want to import graphics files with /includegraphics
\usepackage{mathtools}
\usepackage{graphicx}
\DeclareMathOperator{\Lim}{Lim}
\DeclareMathOperator{\Int}{Int}
\DeclareMathOperator{\Img}{Im}
\DeclareMathOperator{\Cl}{Cl}
\DeclareMathOperator{\A}{\mathcal{A}}
\DeclareMathOperator{\N}{\mathbb{N}}
\DeclareMathOperator{\R}{\mathbb{R}}
\DeclareMathOperator{\Q}{\mathbb{Q}}
\DeclareMathOperator{\bigep}{\mathcal{E}}
\DeclarePairedDelimiter\abs{\lvert}{\rvert}%
\DeclarePairedDelimiter{\norm}{\lVert}{\rVert}

\newcommand*\xoverline[2][0.75]{%
    \sbox{\myboxA}{$\m@th#2$}%
    \setbox\myboxB\null% Phantom box
    \ht\myboxB=\ht\myboxA%
    \dp\myboxB=\dp\myboxA%
    \wd\myboxB=#1\wd\myboxA% Scale phantom
    \sbox\myboxB{$\m@th\overline{\copy\myboxB}$}%  Overlined phantom
    \setlength\mylenA{\the\wd\myboxA}%   calc width diff
    \addtolength\mylenA{-\the\wd\myboxB}%
    \ifdim\wd\myboxB<\wd\myboxA%
       \rlap{\hskip 0.5\mylenA\usebox\myboxB}{\usebox\myboxA}%
    \else
        \hskip -0.5\mylenA\rlap{\usebox\myboxA}{\hskip 0.5\mylenA\usebox\myboxB}%
    \fi}

% info for header block in upper right hand corner
\name{Ravi Raju}
\class{MA 521}
\assignment{Homework \#10}
\duedate{4/25/2019}

\begin{document}

\problemlist{Chapter 7: 4.3, 4.6, 4.7 \\
            Chapter 8: 1.5, 1.11, 1.12, 1.17}

\begin{problem}[Exercise 4.3]
Let $B = \{0\}\cup \{\frac{-1}{n^{2}}\}_{n\in\N}$ and $E=\R\setminus B.$ Consider the series $$\sum_{n=1}^{\infty} \frac{1}{1 + n^{2}x}$$ on the set $E$.
\begin{enumerate}
    \item Prove that the series converges absolutely for all $x \in E$; therefore it converges pointwise to a function $f : E\rightarrow \R$.
    \item Prove that the series converges uniformly to $f$ on $(-\infty, -\delta) \cup (\delta, \infty)\setminus B$ for any $\delta > 0$, but that it does not converge uniformly to $f$ on $E$.
    \item Prove that $f$ is continous. 
    \item Prove that $f(0+) = +\infty$, that therefore $f$ is not a bounded function.
\end{enumerate}
\end{problem}

\begin{solution}
\begin{enumerate}
    \item Consider the series, $A=\sum_{n=1}^{\infty} \frac{1}{xn^{2}}$. For the case that $x > 0$, we know that $\sum_{n=1}^{\infty} \abs{\frac{1}{1 + n^{2}x}}= \sum_{n=1}^{\infty} \frac{1}{1 + n^{2}x} \leq A$. A converges so for this case this series absolutely converges. The other case is $x < 0$. For this case, $1 + n^{2}x < -n^{2}$ after some value $N$. This value can be selected in this fashion: $-n^{2}-n^{2}x \ge 1 \rightarrow -n^{2} \ge \frac{1}{1 + x} \rightarrow n^{2} \ge \abs{\frac{1}{1 + x}}$. So from  $(N, \infty)$ this series absolutely converges by Comparison Test.
    \item From part 1, select a series that converges that is larger than $f$, such as $\sum_{n=1}^{\infty} \frac{1}{1 - \delta n^{2}}= M_n$. This is the largest choice since $\delta$ is the smallest positive and largest negative so it will cover both cases in terms of x. By the Weierstrass $M$-Test, this converges uniformly on the given interval. This is not true for $E$ because $\exists \epsilon > 0$ s.t. $x=\delta - \epsilon$ s.t. $M_n < \frac{1}{\abs{1 - (\delta - \epsilon)n^{2}}}$.
    \item By part 2, we know that $f$ converges uniformly on $(-\infty, -\delta)\cup(\delta, \infty)$ so pick an arbitrary nbd $(a,b) \in (\delta, \infty).$ Uniform Convergence guarantees that $\abs{f(x_1) - f(x_2)} < \epsilon$ for $\abs{x_1 - x_2} <\delta$ which translates directly to the $\epsilon - \delta$ of continutiy. 
    \item Choose the $\sum_{n =1}^{\infty}\frac{1}{n}$. For $x \leq \frac{1}{4}$, choose $n\in\N$ s.t. that $n \ge 2$ then $\frac{1}{1 + xn^{2}} > \frac{1}{n}$ and we know that $\frac{1}{n}$ diverges by p-test so by CST this series diverges as $x\rightarrow0$
\end{enumerate}
\end{solution}

\begin{problem}[Exercise 4.6]
Find the radius of convergence for each of the following power series: $$
\sum_{n=0}^{\infty} n^{n} z^{n} \quad \sum_{n=0}^{\infty} \frac{z^{n}}{n !} \quad \sum_{n=0}^{\infty} z^{n} \quad \sum_{n=0}^{\infty} \frac{z^{n}}{n} \quad \sum_{n=0}^{\infty} \frac{z^{n}}{n^{2}}
.$$
\end{problem}

\begin{solution}
\begin{enumerate}
    \item $\lim_{n \rightarrow \infty}\sup (n^{n})^{\frac{1}{n}} \rightarrow \lim_{n \rightarrow \infty}\sup n\rightarrow +\infty$. So, radius of convergence is zero.
    \item 
    \item 
    \item 
\end{enumerate}
\end{solution}


\begin{problem}[Exercise 4.7]
Consider the power series $\sum_{n=0}^{\infty} c_nz^{n}$. Let $R$ be the radius of convergence of the power series, and assume $R>0$. Let $f: (-R,R) \rightarrow \R$ be the function defined by $f(z)=\sum_{n=0}^{\infty} c_nz^{n}$. Prove the following statements, which refine Thm 4.5.
\begin{enumerate}
    \item For any $r \in (0, R)$, the series $\sum_{n=0}^{\infty}$ converges uniformly on $(-r, r)$ to $f$.
    \item $f$ is continuous on all of $(-R,R)$.
\end{enumerate}
\end{problem}


\begin{solution}
\end{solution}


\begin{problem}[Exercise 1.5]
Let $f: \R \rightarrow \R$ be a function such that $\abs{f(x) - f(y)} \leq (x - y)^{2}$ for all $x, y \in \R.$ Prove that $f$ is constant.
\end{problem}

\begin{solution}

\end{solution}

\begin{problem}[Exercise 1.11]
Let $f: \R \rightarrow \R$ be differentiable, and assume $\lim_{x \rightarrow +\infty}x\abs{f'(x)}=0$. Define a sequence $(a_n)$ in $\R$ by $a_n = f(2n) -f(n)$ for each $n \in \N$. Prove that $a_n \rightarrow$ as $n\rightarrow\infty$.
\end{problem}

\begin{solution}

\end{solution}

\begin{problem}[Exercise 1.12]
Let $f: (a,b) \rightarrow \R$ be differentiable with $f'(x) > 0$ for all $x\in(a,b)$.
\begin{enumerate}
    \item Prove that $f$ is injectibe.
    \item By part (1), there exists a function $g : f((a,b))\rightarrow (a,b)$ such that $g(f(x))= x$ for all $x\in(a, b).$ Prove that $g$ is continuous.
    \item Prove that $g$ is differentiable, and that $g'(f(x))= \frac{1}{f'(x)},$ for all $x \in (a,b)$.
\end{enumerate}
\end{problem}

\begin{solution}

\end{solution}

\begin{problem}[Exercise 1.17]
Use Taylor's Theorem with remainder to estimate $e^{\frac{1}{2}}$ to an accuracy of within $10^{-3}.$ Prove your answer has the desired accuracy.
\end{problem}

\begin{solution}

\end{solution}



\end{document}