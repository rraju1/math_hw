% HMC Math dept HW template example
% v0.04 by Eric J. Malm, 10 Mar 2005
\documentclass[12pt,letterpaper,boxed]{hmcpset}

% set 1-inch margins in the document
\usepackage[margin=1in]{geometry}

% include this if you want to import graphics files with /includegraphics
\usepackage{mathtools}
\usepackage{graphicx}
\DeclareMathOperator{\Lim}{Lim}
\DeclareMathOperator{\Int}{Int}
\DeclareMathOperator{\Img}{Im}
\DeclareMathOperator{\Cl}{Cl}
\DeclareMathOperator{\A}{\mathcal{A}}
\DeclareMathOperator{\N}{\mathbb{N}}
\DeclareMathOperator{\R}{\mathbb{R}}
\DeclareMathOperator{\Q}{\mathbb{Q}}
\DeclareMathOperator{\bigep}{\mathcal{E}}
\DeclarePairedDelimiter\abs{\lvert}{\rvert}%
\DeclarePairedDelimiter{\norm}{\lVert}{\rVert}

\newcommand*\xoverline[2][0.75]{%
    \sbox{\myboxA}{$\m@th#2$}%
    \setbox\myboxB\null% Phantom box
    \ht\myboxB=\ht\myboxA%
    \dp\myboxB=\dp\myboxA%
    \wd\myboxB=#1\wd\myboxA% Scale phantom
    \sbox\myboxB{$\m@th\overline{\copy\myboxB}$}%  Overlined phantom
    \setlength\mylenA{\the\wd\myboxA}%   calc width diff
    \addtolength\mylenA{-\the\wd\myboxB}%
    \ifdim\wd\myboxB<\wd\myboxA%
       \rlap{\hskip 0.5\mylenA\usebox\myboxB}{\usebox\myboxA}%
    \else
        \hskip -0.5\mylenA\rlap{\usebox\myboxA}{\hskip 0.5\mylenA\usebox\myboxB}%
    \fi}

% info for header block in upper right hand corner
\name{Ravi Raju}
\class{MA 521}
\assignment{Homework \#8}
\duedate{4/11/2018}

\begin{document}

\problemlist{Chapter 5: 3.9, 3.10 \\
Chapter 6: 1.9, 4.2, 4.5}

\begin{problem}[Exercise 3.9]
A collection $\A$ of real-valued functions on a set $E$ is said to be \textit{uniformly bounded} on $E$ if there exists $M > 0$ such that $\abs{f(x)} \leq M$ for all $x \in E$, for all $f\in\A.$ (So each function is bounded, and the same bound works for all functions in $\A$.) Let $(f_n)$ be a sequence of bounded functions which converges uniformly to a limit function $f$. Prove that $\{f_n\}$ is uniformly bounded.
\end{problem}

\begin{solution}
$(f_n)$ contains sequence of all bounded functions. By Prop 3.8, $f_n \rightarrow f$ uniformly iff $d_u(f, f_n) = \sup \abs{f_n(x) - f(x)}$ as $n\rightarrow\infty$. So, choose $n\in\N$ s.t. $\max(\abs{f_1(x) - f(x)}, \dots, \abs{f_n(x) - f(x)}, \dots) \forall \, x \in E$. Take this value s.t. $M = \abs{f_n(x) - f(x)} + 1$. By Prop 3.8, this is the largest deviation possible and all other functions will lie in $B(E)$ so they will also be bounded by $M$. So, $\{f_n\}$ is uniformly bounded.
\end{solution}

\begin{problem}[Exercise 3.10]
Let $(f_n)$ and $(g_n)$ be sequences of real-valued functions on a set $E$, which converge uniformly on $E$ to limit functions $f$ and $g$, respectively.
\begin{enumerate}
    \item Prove that $(f_n + g_n)$ converges to $f + g$, uniformly on $E$.
    \item If each $f_n$ and each $g_n$ is bounded, show that $(f_n g_n)$ converges uniformly to $fg$ on $E$.
\end{enumerate} 
\end{problem}

\begin{solution}
\begin{enumerate}
    \item So, for $(f_n + g_n)$ to converge uniformly, we need to show that $\abs{f_n(x) + g_n(x) - f(x) + g(x)} < \epsilon\, \forall \epsilon > 0$. Apply the trianle inequality so $\abs{f_n(x) + g_n(x) - f(x) + g(x)} \leq \abs{f_n(x) - f(x)} + \abs{g_n(x) - g(x)} < \epsilon_1 + \epsilon_2$, where $\epsilon_1$ is the $\sup \abs{f_n - f}$ and $\epsilon_2$ is the $\sup \abs{g_n - g}$. Since $f$, $g$ both converge uniformly on $E$, $f_n + g_n$ is also uniformly converges on $E$.
    \item $(f_n) \leq M$,$(g_n) \leq L, $where $L, M \in \R$, $\abs{g_n(x) - g(x)} < \epsilon_1$, and $\abs{f_n(x) - f(x)} < \epsilon_2$. We need to prove that $\abs{f_n(x)g_n(x) - f(x)g(x)} < \epsilon \forall \, \epsilon > 0.$ So, $\abs{f_n(x)g_n(x) - f(x)g(x)} \leq \abs{f_n(x)}\abs{g_n(x) - g(x)} + \abs{g(x)}\abs{f_n(x) - f(x)} = M\epsilon_1 + L\epsilon_2$. So, $f_n + g_n$ uniformly converges to $fg$ on $E$.
\end{enumerate}
\end{solution}

\begin{problem}[Exercise 1.9]
Prove the second and third points in Prop 1.8.
\end{problem}

\begin{solution}
\begin{enumerate}
    \item For $\rightarrow$, by definition of limits, for all nbd $V$ of $q$, $\exists$ a nbd $U$ of $+\infty$ s.t. $x \in U \cap B\setminus\{+\infty\}\neq\emptyset\rightarrow g(x)\in V.$ So pick $M$ s.t. $(M, +\infty)\subset B.$ So, pick $V = B_{\overline{\R}}(q,\epsilon)$ for some $\epsilon$ in $V$. So, $g(x)\in B_{\overline{\R}}(q,\epsilon) \rightarrow \abs{g(x) - q} < \epsilon$. For the other direction, for every $\epsilon > 0$, $\exists M \in \R$ s.t. $x > M$ and $x \in B$ together imply that $\abs{g(x) - q} < \epsilon.$ $\abs{g(x) - q} < \epsilon \rightarrow g(x) \in B_{\overline{\R}}(q, \epsilon)$ for every $\epsilon > 0.$ Pick an $M$ in $\R$ and set $U$ in $\R$ as  $(M, +\infty)$. So $x \in (M, + \infty)$. So $x\in (M, +\infty)$ and $x\in B$ and $(M,+\infty)\cap B \setminus \{+\infty\}\neq\emptyset.$
    \item For every neighborhood $V$ of $+\infty$, $\exists$ a nbd $U$ of $+\infty$ s.t. $x\in U\cap C \setminus \{+\infty\}\neq\emptyset$. Let $U$ be the neighborhood of $+\infty$ for some $P\in\R$, $(P,+\infty)$. So, let $x\in C$ and $x > P$ to be in $(P, +\infty)$. So, let $h(x)\in V$ around $+\infty$. So, for any choice of $N$, $h(x)$ will always be in $V$. For $\leftarrow$, let $U$ be a neighborhood of $+\infty$ s.t. $x\in (P,+\infty)$ and $x\in C$. $h(x) > N$ implies that $V$ can be chosen as $(N, +\infty)$. To see if $+\infty$ is limit point, $U\cap C \setminus \{+\infty\}\neq \emptyset.$
\end{enumerate}
\end{solution}

\begin{problem}[Exercise 4.2]
Let $(s_n)_{n=1}^{\infty}$ and $(t_n)_{n=1}^{\infty}$ be sequences in $\overline{\R}$ and let $(u_n)_{n=1}^{\infty}$ be a sequence in $\R$. Prove the following statements.
\vspace{-2mm}
    \begin{enumerate}
        \itemsep0em
        \item If $s_n \leq t_n$ for each $n\in\N$ and $\lim_{n\rightarrow \infty} s_n = + \infty$, then $\lim_{n\rightarrow \infty} t_n = + \infty$ as well.
        \item If $(s_n)$ and $(t_n)$ converge in $\overline{\R}$ to $s$ and $t$, respectively, and if $s_n \leq t_n$ for each $n\in\N,$ then $s \leq t.$
    \end{enumerate}
\end{problem}

\begin{solution}
\begin{enumerate}
        \itemsep0em
        \item 
        \item
    \end{enumerate}
\end{solution}

\begin{problem}[Exercise 4.2]
Let $(a_n)_{n=1}^{\infty}$ and $(b_n)_{n=1}^{\infty}$ be sequences in of real numbers. Prove that $$\lim_{n \rightarrow} \sup(a_n + b_n) \leq \lim_{n \rightarrow \infty} \sup(a_n) + \lim_{n \rightarrow \infty} \sup(b_n),$$ provided that the RHS isn't of the form $\infty - \infty$.
\end{problem}

\begin{solution}

\end{solution}

\end{document}
