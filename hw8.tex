% HMC Math dept HW template example
% v0.04 by Eric J. Malm, 10 Mar 2005
\documentclass[12pt,letterpaper,boxed]{hmcpset}

% set 1-inch margins in the document
\usepackage[margin=1in]{geometry}

% include this if you want to import graphics files with /includegraphics
\usepackage{mathtools}
\usepackage{graphicx}
\DeclareMathOperator{\Lim}{Lim}
\DeclareMathOperator{\Int}{Int}
\DeclareMathOperator{\Img}{Im}
\DeclareMathOperator{\Cl}{Cl}
\DeclareMathOperator{\A}{\mathcal{A}}
\DeclareMathOperator{\N}{\mathbb{N}}
\DeclareMathOperator{\R}{\mathbb{R}}
\DeclareMathOperator{\Q}{\mathbb{Q}}
\DeclareMathOperator{\bigep}{\mathcal{E}}
\DeclarePairedDelimiter\abs{\lvert}{\rvert}%
\DeclarePairedDelimiter{\norm}{\lVert}{\rVert}

\newcommand*\xoverline[2][0.75]{%
    \sbox{\myboxA}{$\m@th#2$}%
    \setbox\myboxB\null% Phantom box
    \ht\myboxB=\ht\myboxA%
    \dp\myboxB=\dp\myboxA%
    \wd\myboxB=#1\wd\myboxA% Scale phantom
    \sbox\myboxB{$\m@th\overline{\copy\myboxB}$}%  Overlined phantom
    \setlength\mylenA{\the\wd\myboxA}%   calc width diff
    \addtolength\mylenA{-\the\wd\myboxB}%
    \ifdim\wd\myboxB<\wd\myboxA%
       \rlap{\hskip 0.5\mylenA\usebox\myboxB}{\usebox\myboxA}%
    \else
        \hskip -0.5\mylenA\rlap{\usebox\myboxA}{\hskip 0.5\mylenA\usebox\myboxB}%
    \fi}

% info for header block in upper right hand corner
\name{Ravi Raju}
\class{MA 521}
\assignment{Homework \#8}
\duedate{4/11/2018}

\begin{document}

\problemlist{Chapter 5: 3.9, 3.10 \\
Chapter 6: 1.9, 4.2, 4.5}

\begin{problem}[Exercise 3.9]
A collection $\A$ of real-valued functions on a set $E$ is said to be \textit{uniformly bounded} on $E$ if there exists $M > 0$ such that $\abs{f(x)} \leq M$ for all $x \in E$, for all $f\in\A.$ (So each function is bounded, and the same bound works for all functions in $\A$.) Let $(f_n)$ be a sequence of bounded functions which converges uniformly to a limit function $f$. Prove that $\{f_n\}$ is uniformly bounded.
\end{problem}

\begin{solution}
$(f_n)$ contains sequence of all bounded functions. By Prop 3.8, $f_n \rightarrow f$ uniformly iff $d_u(f, f_n) = \sup \abs{f_n(x) - f(x)}$ as $n\rightarrow\infty$. So, choose $n\in\N$ s.t. $\max(\abs{f_1(x) - f(x)}, \dots, \abs{f_n(x) - f(x)}, \dots) \forall \, x \in E$. Take this value s.t. $M = \abs{f_n(x) - f(x)} + 1$. By Prop 3.8, this is the largest deviation possible and all other functions will lie in $B(E)$ so they will also be bounded by $M$. So, $\{f_n\}$ is uniformly bounded.
\end{solution}

\begin{problem}[Exercise 3.10]
Let $(f_n)$ and $(g_n)$ be sequences of real-valued functions on a set $E$, which converge uniformly on $E$ to limit functions $f$ and $g$, respectively.
\begin{enumerate}
    \item Prove that $(f_n + g_n)$ converges to $f + g$, uniformly on $E$.
    \item If each $f_n$ and each $g_n$ is bounded, show that $(f_n g_n)$ converges uniformly to $fg$ on $E$.
\end{enumerate} 
\end{problem}

\begin{solution}

\end{solution}

\begin{problem}[Exercise 1.9]
Prove the second and third points in Prop 1.8.
\end{problem}

\begin{solution}

\end{solution}

\begin{problem}[Exercise 4.2]
Let $(s_n)_{n=1}^{\infty}$ and $(t_n)_{n=1}^{\infty}$ be sequences in $\overline{\R}$ and let $(u_n)_{n=1}^{\infty}$ be a sequence in $\R$. Prove the following statements.
\vspace{-2mm}
    \begin{enumerate}
        \itemsep0em
        \item If $s_n \leq t_n$ for each $n\in\N$ and $\lim_{n\rightarrow \infty} s_n = + \infty$, then $\lim_{n\rightarrow \infty} t_n = + \infty$ as well.
        \item If $(s_n)$ and $(t_n)$ converge in $\overline{\R}$ to $s$ and $t$, respectively, and if $s_n \leq t_n$ for each $n\in\N,$ then $s \leq t.$
    \end{enumerate}
\end{problem}

\begin{solution}
\begin{enumerate}
        \itemsep0em
        \item 
        \item
    \end{enumerate}
\end{solution}

\begin{problem}[Exercise 4.2]
Let $(a_n)_{n=1}^{\infty}$ and $(b_n)_{n=1}^{\infty}$ be sequences in of real numbers. Prove that $$\lim_{n \rightarrow} \sup(a_n + b_n) \leq \lim_{n \rightarrow \infty} \sup(a_n) + \lim_{n \rightarrow \infty} \sup(b_n),$$ provided that the RHS isn't of the form $\infty - \infty$.
\end{problem}

\begin{solution}

\end{solution}

\end{document}
