% HMC Math dept HW template example
% v0.04 by Eric J. Malm, 10 Mar 2005
\documentclass[12pt,letterpaper,boxed]{hmcpset}

% set 1-inch margins in the document
\usepackage[margin=1in]{geometry}

% include this if you want to import graphics files with /includegraphics
\usepackage{mathtools}
\usepackage{graphicx}
\DeclareMathOperator{\Lim}{Lim}
\DeclareMathOperator{\Int}{Int}
\DeclareMathOperator{\Img}{Im}
\DeclareMathOperator{\Cl}{Cl}
\DeclareMathOperator{\A}{\mathcal{A}}
\DeclareMathOperator{\N}{\mathbb{N}}
\DeclareMathOperator{\R}{\mathbb{R}}
\DeclareMathOperator{\Q}{\mathbb{Q}}
\DeclareMathOperator{\bigep}{\mathcal{E}}
\DeclarePairedDelimiter\abs{\lvert}{\rvert}%
\DeclarePairedDelimiter{\norm}{\lVert}{\rVert}

\newcommand*\xoverline[2][0.75]{%
    \sbox{\myboxA}{$\m@th#2$}%
    \setbox\myboxB\null% Phantom box
    \ht\myboxB=\ht\myboxA%
    \dp\myboxB=\dp\myboxA%
    \wd\myboxB=#1\wd\myboxA% Scale phantom
    \sbox\myboxB{$\m@th\overline{\copy\myboxB}$}%  Overlined phantom
    \setlength\mylenA{\the\wd\myboxA}%   calc width diff
    \addtolength\mylenA{-\the\wd\myboxB}%
    \ifdim\wd\myboxB<\wd\myboxA%
       \rlap{\hskip 0.5\mylenA\usebox\myboxB}{\usebox\myboxA}%
    \else
        \hskip -0.5\mylenA\rlap{\usebox\myboxA}{\hskip 0.5\mylenA\usebox\myboxB}%
    \fi}

% info for header block in upper right hand corner
\name{Ravi Raju}
\class{MA 521}
\assignment{Homework \#7}
\duedate{3/28/2018}

\begin{document}

\problemlist{1.2, 2.2, 2.9, 2.10, 2.11, 2.23,
2.32, 2.36, 2.37.}

\begin{problem}[Exercise 1.2]
Let $(X, d_X)$ and $(Y, d_Y)$ be metric spaces, and let $E$ be a subset of $X$. Let $f : E \rightarrow Y$ be a function, and let $p$ be a limit point of $E$ in $X$. Prove that $f(x) \rightarrow q$ as $x\rightarrow p$ if and only if for every $\epsilon > 0$, there exists $\delta > 0$ such that $x \in E$ and $0 < d_X(x,p) < \delta$ imply together that $d_Y(f(x), q) < \epsilon.$
\end{problem}

\begin{solution}

\end{solution}

\begin{problem}[Exercise 2.2]
Let $(X, d_X)$ and $(Y, d_Y)$ be metric spaces; let $f : X \rightarrow Y$ be a function. Prove that $f$ is continuous at $p\in X$ if and only if for every $\epsilon$ > 0, there exists $\delta > 0$ such that $x \in B_X(p,\delta)$ implies $f(x)\in B_Y(f(p), \epsilon)$.
\end{problem}

\begin{solution}

\end{solution}

\begin{problem}[Exercise 2.9]
Assume $f: \R \rightarrow \R$ is a function satisfying $\lim_{h\rightarrow 0} [f(x + h) - f(x - h)] = 0,$ for all $x \in \R$. Does it follow that $f$ must be continuous? If so, give a proof; if not, give a counterexample.
\end{problem}

\begin{solution}

\end{solution}

\begin{problem}[Exercise 2.10]
Let $(X, d_X)$ and $(Y, d_Y)$ be metric spaces and $f: X \rightarrow Y$ a function. 
\vspace{-2mm}
    \begin{enumerate}
        \itemsep0em
        \item Show that $f$ is continuous if and only if $f^{-1}(C)$ is closed on $X$ whenever $C$ is closed in $Y$.
        \item Show that $f: X \rightarrow Y$ is continuous if and only if $f(\overline{A})\subset\overline{f(A)}$ for every subset $A$ of $X$.
        \item Consider the (continuous) function $g: \R \rightarrow \R$ given by $g(x) = \frac{1}{1 + x^{2}}.$ Give an example of a subset $A$ of $\R$ such that $g(\overline{A})\neq\overline{g(A)}.$
    \end{enumerate}
\end{problem}

\begin{solution}

\end{solution}

\begin{problem}[Exercise 2.11]
Let $(X, d_X)$ and $(Y, d_Y)$ be metric spaces and let $f$ and $g$ be continuous functions from $X$ to $Y$. Assume $E$ is a dense subset of $X$.
\vspace{-2mm}
\begin{enumerate}
    \itemsep0em
    \item Prove that $f(E)$ is dense in $f(X).$ (Hint: Use Exercise 1.30) in Chapter 4 and Exercise 2.10 above.)
    \item Prove that if $f(x) = g(x)$ for all $x \in E$, then $f(x) = g(x)$ for all $x\in X$.
\end{enumerate} 
\end{problem}

\begin{solution}
\begin{enumerate}
     \itemsep0em
    \item By 2.10, $f(\overline{E})\subset\overline{f(E)}$. We want to prove that for an open subset $f(C)$ of $Y$ that $f(E) \cap f(C)\neq \emptyset$. So, $Y\setminus f(C)$ is a closed set in $Y$ then also $f^{-1}(Y\setminus f(C))$ is a closed set in $X$. $f^{-1}(Y\setminus f(C))=X\setminus f^{-1}(f(C))$ is closed so $f^{-1}(f(C))$ is open in $X$. So, $E\cap f^{-1}(f(C))\neq \emptyset.$ So there exists an element$x\in E$ s.t. $x \in f^{-1}(f(C))$ so $f(x) \in f(C).$ So, $f(x) \in f(E) \rightarrow f(C)\cap f(E)\neq \emptyset$. So $f(E)$ is dense in $f(X).$
    \item Assume by contradiction that $f(a)\neq g(a), a\in X$. Let $d(f(a), g(a)) = r > 0$. Since $f$ is continuous at $a$ so $\exists \delta_1 > 0$ s.t. $f(B(a, \delta_1))\subset B(f(a),\frac{r}{4})$. $g$ is continuous at $a$ so $\exists \delta_2 > 0$ s.t. $g(B(a, \delta_2))\subset B(g(a),\frac{r}{4})$. Take $\delta = \min(\delta_1, \delta_2)$. Then, $f(B(a, \delta))\subset B(f(a),\frac{r}{4})$ and $g(B(a, \delta))\subset B(g(a),\frac{r}{4})$. Since $E$ is dense in $X$ so $B(a, \delta)\cap E \neq \emptyset$. Take $k\in B(a, \delta)\cap E$. Then, $f(k)=g(k).$ So, $f(k)\in B(f(a), \frac{r}{4})$ and $g(k)\in B(g(a), \frac{r}{4})$. Hence, by triangle inequality, we have $d(f(a), g(a)) \leq d(f(a), f(k)) + d(f(k), g(k)) + d(g(k), g(a)) < \frac{2r}{4}$, which is false since $d(f(a), g(a))=r.$ So, $f(x)=g(x)$ for all $x\in X$
\end{enumerate}
\end{solution}

\begin{problem}[Exercise 2.23]
\begin{enumerate}
    \itemsep0em
    \item Find a closed subset of $E$ of $\R$ and a continuous function $f: \R \rightarrow \R$ is continuous such that $f(E)$ is not closed.
    \item Find a bounded subset $E$ of $\R$ and a continuous function $f : E \rightarrow \R$ such that $f(E)$ is not bounded.
    \item Show that if $E$ is a bounded subset of $\R$ and $f: \R \rightarrow \R$ is continuous, then $f(E)$ is bounded.
\end{enumerate} 
\end{problem}

\begin{solution}
\begin{enumerate}
    \itemsep0em
    \item Consider the function $f = \arctan(x)$ on the closed interval $[0, \infty)$. $f$ is continuous but does not yield a closed set as $f(0)=0$ and $f(\infty)=\frac{\pi}{2}$ so the interval was $[0, \frac{\pi}{2})$.
    \item Let $E$ be the set $[0, 1]$ for $\frac{1}{x}$ defined on E. $\nexists$ any $r$ such that any ball in $\R$ can contain $\frac{1}{0}$.
    \item For some $\delta > 0$, $E\subset B_X(p, \delta)$ since $E$ is bounded for some $p\in X$. So by definition of continuity of $f$, $f(B_X(p, \delta))\subset B_Yf(p), \epsilon)$ so $f(E)$ is bounded.
\end{enumerate} 
\end{solution}

\begin{problem}[Exercise 2.32]
Prove that the set $R^{2}\setminus\{0,0\}$ is path-connected, and therefore connected. Then, use the function $\frac{x}{\abs{x}}$ to show that $S = \{x\in\R^{2}:\abs{x} = 1\}$ is connected.
\end{problem}

\begin{solution}
Let $a_0, a_1 \in \R^{2}\setminus\{0,0\}$. So, use polar coordinates for a curve: $a_i = r_i(\cos \theta_i, \sin \theta_i)$ with $r_i > 0$ and let $f(t) = r(t)(\cos \theta(t), \sin \theta(t))$ where $r(t) = (1 - t)r_0 + tr_1$ and $\theta(t)=(1-t)\theta_0 + t\theta_1$.
Then $\theta(0) = \theta_0, \theta(1)=\theta_1, r(0) = r_0, r(1)=r_1 \rightarrow f(0)=a_0$ and $f(1) = a_1$. $\forall \, t\in [0,1], r(t) > 0$ since $t, 1-t, r_0, r_1 > 0$. So, $f(t)\neq 0 \forall t$ and therefore $f$ defines path from $a_0$ to $a_1$. So, $\R^{2}\setminus\{0,0\}$ is path-connected and is connected. \\
Set $f(t)= \frac{a(1 - t) + tb}{\abs{a(1 - t) + tb}}$. Since $\abs{a(1 - t) + tb} \in S$, this will always be constrained to 1. So then $f(0) = a, f(1) = b$ and $S$ is path-connected and thus connected.
\end{solution}

\begin{problem}[Exercise 2.40]
Assume $f: X \rightarrow Y$ and $g: Y \rightarrow Z$ are uniformly continuous functions, where $(X, d_X), (Y, d_Y),$ and $(Z, d_Z)$ are metric spaces. Prove that $g \circ f$ is uniformly continuous.
\end{problem}

\begin{solution}
Take $(x, y)$ in $X$ s.t. $d_X(x, y) < \delta (\delta > 0)$ then $d_Y(f(x), f(y)) < \epsilon.$ Set $\delta_1 = \epsilon$ so that $d_Y(f(x), f(y))<\delta_1$ then $d_Z(g(f(x)),g(f(y))) < \epsilon_2$.
So for any choice of arbitrary $\epsilon_2$ so $g \circ f$ is uniformly continuous.
\end{solution}


\begin{problem}[Exercise 2.41]
Let $E$ be a bounded subset of $\R^{k}$ and let $f : E \rightarrow \R$ be a uniformly continuous function. Show that $f$ is bounded. (Hint: You will need to use compactness of $\overline{E}$ at some point.)
\end{problem}

\begin{solution}
$\overline{E}$ is closed and bounded so it is compact. Because $\overline{E}$ is compact, it can be written as $\bigcup_{i = 1}^{n} B_{R^{k}}(x_i, \delta)$ which is a finite open cover. Uniform continuity of $f$ states that for $d(x,y) < \delta$ in $x, y \in \R^{k}$ then $d(f(x), f(y)) < \epsilon.$ Get the max distance of all $f(x_i)$ with each other and add $\epsilon$ and set this quantity to $r$. Then choose any $f(x_i)$ as center as $B_X(f(x_i), r)$ so $f$ is bounded. 
\end{solution}



\end{document}
