% HMC Math dept HW template example
% v0.04 by Eric J. Malm, 10 Mar 2005
\documentclass[12pt,letterpaper,boxed]{hmcpset}

% set 1-inch margins in the document
\usepackage[margin=1in]{geometry}

% include this if you want to import graphics files with /includegraphics
\usepackage{mathtools}
\usepackage{graphicx}
\DeclareMathOperator{\Lim}{Lim}
\DeclareMathOperator{\Int}{Int}
\DeclareMathOperator{\Cl}{Cl}
\DeclareMathOperator{\N}{\mathbb{N}}
\DeclarePairedDelimiter\abs{\lvert}{\rvert}%

% info for header block in upper right hand corner
\name{Ravi Raju}
\class{MA 521}
\assignment{Homework \#4}
\duedate{2/28/2018}

\begin{document}

\problemlist{1.9, 1.10, 1.13, 1.14, 1.15, 1.21, 1.24}

\begin{problem}[Exercise 1.9]
Let $E_1$ and $E_2$ be subsets of a metric space $(X,d).$ Prove that $$\Lim_X(E_1 \cup E_2) = \Lim_X(E_1)\cup\Lim_X(E_2).$$
\end{problem}

\begin{solution}
For $\subset$, assume $x\in \Lim_X(E_1\cup E_2)$ and $U$ is a neighborhood of $x$ in $X$. So, $\emptyset\neq U \cap (E_1 \cup E_2)\setminus \{x\}.$ $U \cap [(E_1\setminus\{x\})\cup(E_2\setminus\{x\})] = [U\cap(E_1\setminus\{x\})\cup U \cap (E_2\setminus\{x\})] = \Lim_X(E_1)\cup\Lim_X(E_2).$ For $\supset$, using Proposition 1.8, $E_1\subset E_1 \cup E_2 \rightarrow \Lim_X(E_1)\subset\Lim_X(E_1\cup E_2).$ Similarly, $E_2\subset E_1 \cup E_2 \rightarrow \Lim_X(E_2)\subset\Lim_X(E_1\cup E_2).$ So, $\Lim_X(E_1) \cup \Lim_X(E_2) \subset \Lim_X(E_1\cup E_2).$ So, $\Lim_X(E_1 \cup E_2) = \Lim_X(E_1)\cup\Lim_X(E_2).$
\end{solution}

\begin{problem}[Exercise 1.10]
Let $(X,d)$ be a metric space, and assume $E\subset Y \subset X.$ Prove that $$\Lim_Y(E) = \Lim_X(E)\cap Y.$$
\end{problem}

\begin{solution}
For $\subset$, $\Lim_Y(E) = V \cap (E\setminus\{x\}) = (U \cap Y)\cap(E\setminus\{x\})$, where $V$ is open in $Y$ and $U$ is open in X (applied Theorem 2.19). So, $(U \cap Y)\cap(E\setminus\{x\}) \rightarrow [U \cap E\setminus\{x\}]\cap Y = \Lim_X(E)\cap Y.$ For $\supset$, $\Lim_X(E)\cap Y \rightarrow [V \cap (E\setminus\{x\})]\cap Y$, where $V$ is an open set neighborhood of $x$ with respect to $X$. So, $[(V \cap Y)\cap[E\setminus\{x\}\cap Y]]$. Applying Theorem 2.19 to $V\cap Y = U$ where $U$ is open in $Y$. Then, $U\cap[E\setminus\{x\}\cap Y] = U \cap [E\setminus\{x\}]$, since $E \subset Y$. So, $U \cap [E\setminus\{x\}] \rightarrow \Lim_Y(E).$ So, $Lim_Y(E) = \Lim_X(E)\cap Y$.
\end{solution}


\begin{problem}[Exercise 1.13]
Let $(X,d)$ be a metric space, and assume $Y \subset X.$ Let $(x_n)_{n=1}^{\infty}$ be a sequence in $Y$ and let $x$ be a point of $X$. Prove that the following two statements are equivalent:
\vspace{-2mm}
\begin{enumerate}
	\itemsep0em
	\item $x_n \rightarrow x$ in $X$, and $x\in Y.$
	\item $x_n \rightarrow x$ in $Y$.
\end{enumerate}
\end{problem}

\begin{solution}
For $\subset$, if $x_n \rightarrow x$ in $X,$ and $x \in Y,$ then for some neighborhood $V$ of $x$ and $\exists$ some $n \in \N$ s.t. $x_n\in V.$ By Theorem 2.19, $U=V\cap Y \implies U$ is open in $Y$ so $x_n \in U.$ This implies that $x_n \rightarrow x$ and $x\in Y$. For $\supset$, $\exists$ a neighborhood $V$ around $x$ with respect to $Y$ s.t. $n \in \N$ that $x_n\in V.$ $V\subset Y \subset X$ so $x\in Y.$ If $x_n \in V,$ then $V = U \cap Y$ where $x_n \in U.$ So, $x_n\rightarrow x$ in $X$. 
\end{solution}

\begin{problem}[Exercise 1.14]
Let $(X,d)$ be a metric space, and let $(x_n)_{n=1}^{\infty}$ be a sequence in $Y$ and let $x$ be a point of $X$. Prove that the following statements are equivalent:
\vspace{-2mm}
\begin{enumerate}
	\itemsep0em
	\item $x_n \rightarrow x$ in $X$
	\item For every $\epsilon > 0,$ there exists $N\in\N$ such that $n \ge N$ implies $x_n\in B_X(x,\epsilon) ($i.e. $ d(x, x_n) < \epsilon).$
	\item $d(x, x_n) \rightarrow 0$ as $n\rightarrow\infty$.
\end{enumerate}
\end{problem}

\begin{solution}
Let's consider the first two points first. For $x_n \rightarrow x$ in $X$, $\exists$ a neighborhood $U$ around $x$ in $X$ s.t. $n\in \N$ that $x_n\in U$ and $n\ge N$, where $N\in\N$. For $\subset$, set $U$ to the ball $B_X(x, \epsilon)$ for $\epsilon > 0$ so $x_n \in B_X(x, \epsilon)$. For $\supset$, use the same reasoning that $n \ge N$ implies $B_X(x, \epsilon)$. $B_X(x, \epsilon)$ is a neighborhood of $x$ which is open in $X$. Since $x_n \in B_X(x, \epsilon)$ for $n\ge N$, so $x_n \rightarrow x$ in $X.$ For the last equality, let us look at points 2 and 3. To prove the direction $(2) \rightarrow (3)$, let's argue via proof by contrapositive. Assume that $d(x, x_n) \rightarrow \zeta$ as $n\rightarrow\infty$ for some $n \in \N$. If this is true, then $\exists$ some $\epsilon > 0$ s.t. $\zeta > \epsilon$ so then $x_n \notin B_X(x, \epsilon).$ So, for $x_n$ to be in the set $B_X(x, \epsilon)$, $d(x, x_n)$ must tend to 0 as $n\rightarrow \infty.$ For the other inclusion $\supset$, for any $\epsilon$ selected, there will always be another $n \in \N$ such that $\epsilon > d(x,x_n)$ so $x_n\in B_X(x, \epsilon).$ 
\end{solution}


\begin{problem}[Exercise 1.15]
Let $(s_n)_{n=1}^{\infty}$ and $(t_n)_{n=1}^{\infty}$ be sequences of real numbers, with $t_n > 0$ for each $n\in\N.$ Assume that $t_n \rightarrow 0$ as $n\rightarrow \infty.$
\vspace{-2mm}
\begin{itemize}
	\itemsep0em
	\item Prove that if $\abs{s_n - s} < t_n$ for all $n \in \N$, then $s_n \rightarrow s$ as $n \rightarrow \infty.$
	\item Prove that if $\frac{1}{n}\rightarrow 0$ as $n\rightarrow \infty.$
\end{itemize}
\end{problem}

\begin{solution}
\vspace{-2mm}
\begin{itemize}
	\itemsep0em
	\item Given $\abs{s_n - s} < t_n \forall n$ is true, for any $t_n > 0,$ $\exists N \in \N$ s.t. $n \ge N$ implies $s_n \in B_X(s, t_n)$ as $d(s_n , s) < t_n \forall n\in\N$ so $s_n\rightarrow s$ as $n \rightarrow \infty.$
	\item Let $a_n = (\frac{1}{n})_{n=1}^{\infty}$. Clearly, $\forall n\in \N, a_{n+1} < a_{n}$ (this is the Archimedean property). If $a_N < \epsilon$ for $N \in \N$ for any $\epsilon > 0$, $\exists n \ge N$ s.t. $a_n < a_N < \epsilon$ so $a_n \in B_X(0, \epsilon).$ 
\end{itemize}
\end{solution}

\begin{problem}[Exercise 1.21]
Let $(X,d)$ be a metric space, and let $E$ be a subset of $X$. Prove that $\Lim_X(E)$ is a closed set of $X$.
\end{problem}

\begin{solution}
Let $D = \Lim_X(E).$ To show that $D$ is closed, we need to prove $\Lim_X(D)\subset D.$ For all neighborhoods $U$ centered around $x$, $\emptyset\neq U \cap (D\setminus \{x\})\rightarrow \exists y \in \Lim_X(D), y \in U.$ For all neighborhoods $V$ centered around $y$, $\emptyset\neq V \cap (D\setminus \{y\})\rightarrow \exists z \in D.$ 
\end{solution}

\begin{problem}[Exercise 1.24]
Let $(X,d)$ be a metric space, and let $E$ be a subset of $X$. Prove that $$X\setminus\Cl_X(E)=\Int_X(X\setminus E)$$. 
\end{problem}

\begin{solution}

\end{solution}

\end{document}