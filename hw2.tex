% HMC Math dept HW template example
% v0.04 by Eric J. Malm, 10 Mar 2005
\documentclass[12pt,letterpaper,boxed]{hmcpset}

% set 1-inch margins in the document
\usepackage[margin=1in]{geometry}

% include this if you want to import graphics files with /includegraphics
\usepackage{graphicx}

% info for header block in upper right hand corner
\name{Ravi Raju}
\class{MA 521}
\assignment{Homework \#2}
\duedate{2/6/2019}

\begin{document}

\problemlist{Chapter 1: 4.18, 4.19, 4.22; \\ Chapter 2: 1.6, 1.7, 2.3, 2.4, 2.5, 3.3, 4.6, 4.8, 4.9, 4.10}

\begin{problem}[Exercise 4.18.]
Let $A$ and $B$ be sets. Assume $A$ is infinite, $B$ is countable, and $A$ and $B$ are disjoint. Prove $A \sim A \cup B.$ Hint: The strategy of Theorem 4.16 may be useful. 
\end{problem}


\begin{solution}
If $A$ is infinite, we have $C\subset A$, a countably infinite set. By Proposition 4.21, the union of two countable sets is still countable, $B\cup C$, which is countably infinite. Since $((A\cup B)\setminus B\cup C) \cap C$ and $((A\cup B)\setminus B\cup C) \cap (B\cup C)$ are both empty, $A \cup B = ((A\cup B)\setminus B\cup C) \cup (B\cup C) \sim ((A\cup B)\setminus B\cup C) \cup C = A.$
\end{solution}

\begin{problem}[Exercise 4.19.]
Let $X$ and $Y$ be sets. Assume $Y$ is countable and $X\setminus Y$ is infinite. Prove that $X \sim X \cup Y \sim X \setminus Y.$ Hint: Each of the equivalences can be done extremely quickly if you use the previous exercise and some set manipulations.
\end{problem}

\begin{solution}
If $X\setminus Y$ is infinite, $X \setminus Y$ must have a countably infinite subset. This means that $X$ must be infinite. We can use Exercise 4.18 but we need to prove that $X$ and $Y$ are disjoint sets. Let's solve by contradiction.
\\ \\
Assume $a_1 \in X, Y$ s.t. $X \cap Y = \{a_1\}.$ This means that $X \setminus Y$ will be a proper subset of $X$. We can apply Theorem 4.16 to say $X \sim X\setminus Y.$ But then, $X \cap Y \sim (X \cap Y)\cap Y = \emptyset$, which is a contradicts our assumption. This suggest that $X$ and $Y$ are disjoint and apply Exercise 4.18 directly to say $X \sim X \cup Y \sim X \setminus Y.$
\end{solution}

\begin{problem}[Exercise 4.22.]
Let $X$ be a countable set. 
\vspace{-2mm}
\begin{enumerate}
	\itemsep0em
	\item Prove inductively that $X^{n}\sim X^{n-1} \times X$ for any $n \in \mathbb{N}$.
	\item Prove inductively that $X^{n}$ is countable for any $n \in \mathbb{N}$.
\end{enumerate}
\end{problem}
\begin{solution}
\begin{enumerate}
	\itemsep0em
	\item WLOG, let $n=2.$ For the base case, by definition of n-tuples, $X^{2}=X\times X \sim X^{1}\times X$. For the inductive step, assume statement is true for $n$, $X^{n+1}=(X \times X \times \dots)\times X = X^{n}\times X = X^{(n + 1)-1}\times X.$
	\item WLOG, let $n=2.$ $X^{2}=X\times X=\{ (a,b) : a \in X$ and $ b \in X \}$. If $X \cup X$ is countable by Proposition 4.21, then $X\times X$ should also be countable. For the inductive step, let $n=k + 1$ assume $X^{k}$ is countable. $X^{k+1}=X \times X^{k}\implies$ $X$ is countable and $X^{k}$ is countable by assumption so by Proposition 4.21, $X^{k+1}$ should be countable.
\end{enumerate}
\end{solution}


\begin{problem}[Exercise 1.6.]
Let $E, F,$ and $G$ be nonempty subsets of an ordered set $(S,\le).$ Prove the following statements. 
\vspace{-2mm}
\begin{enumerate}
	\itemsep0em
	\item If $\alpha\in S$ is a lower bound for $E$ and $\beta\in S$ is an upper bound for $E$, then $\alpha \le \beta$.
	\item $\sup E \leq \inf F$ if and only if $x \leq y$ for any $x\in E, y \in F.$
	\item If $E \subset G,$ then $\sup E \leq \sup G.$
\end{enumerate}
\end{problem}
\begin{solution}
\vspace{-2mm}
\begin{enumerate}
	\itemsep0em
	\item By definition of upper bound, $\forall x \in E : x \leq \beta$. By definition of lower bound, $\forall x \in E : x \ge \alpha.$ So, $\alpha \leq x \leq \beta \implies \alpha \leq E \leq \beta \implies \alpha \leq \beta.$
	\item 
	\begin{enumerate}
		\itemsep0em
		\item Let us prove this $\rightarrow$ direction first. Given $\sup E \leq \inf F$. Let's solve by contradiciton. Assume $x > y$ for any $x \in E, y \in F$ . Say $\beta_1 = \sup E$, implying $\beta_1$ is an upper bound for $E$. So by definition, $x < \beta_1 \, \forall x\in E.$ Say $\alpha_1 = \inf F$, implying $\alpha_1$ is a lower bound for $F$. So by definition, $\alpha_1 \le y \, \forall y\in F.$ By the given statement, $\beta_1\leq \alpha_1 \implies x \leq \beta_1 \leq \alpha_1 \leq y$. This establishes a contradiction so $x \leq y$.
		\item Now the other direction, $\leftarrow$. Given $x \leq y$ for any $x \in E, y \in F$. Let $\beta_2$ be the upper bound for $E$. Let $\alpha_2$ be the upper bound for $F$. $x \leq \beta_2 \leq \alpha_2\leq y$; the tightest bounds for this expression would be if $\beta_2=\sup E$ and $\alpha_2=\inf F.$ $x \leq \sup E \leq \inf F \leq y \implies \sup E \leq \inf F.$
	\end{enumerate}	
	\item Let $a = \sup G$ and $b = \sup E$. Assume $b > a.$ If $b$ is larger than $a$, $a$ could not be the upper bound of $G$ since $E \subset G.$ So, this establishes a contradiction and $\sup E \leq \sup G.$ 
\end{enumerate}
\end{solution}

\begin{problem}[Exercise 1.7.]
Let $(S,\leq)$ be an ordered set, let $f$ and $g$ be functions from $X$ to $S$ and let $A$ be a subset of $X$. Assume that $f(x)\leq g(x)$ for all $x \in A$, and that furthermore $\sup_A f$ and $\sup_A g$ exist in $S$. Prove that $\sup_A f \leq \sup_A g.$
 \end{problem}

\begin{solution}
Given $\sup_{A}f=\sup\{f(x) : x \in A\}=\beta, \sup_{A}g=\sup\{g(x) : x \in A\}=\alpha,$ and $f(x) \leq g(x)\, \forall x \in A$. Clearly, since $\beta$ is an upper bound for $f$, $f(x)\leq \beta \leq g(x) \forall x\in A.$ Since $\alpha$ is an upper bound for $g$, $f(x)\leq \beta \leq g(x) \leq \alpha\, \forall x\in A \implies \beta \leq \alpha = \sup_A f \leq \sup_A g. $ 
\end{solution}

\begin{problem}[Exercise 2.3.]
Let $A$ be a nonempty subset of an ordered field $(F,+,\cdot,\leq)$. Assume that $\sup A$ and $\inf A$ exist in $F$, and let $c$ be any element of $F$. Define the set $cA := \{ ca : a \in A\}.$

\vspace{-2mm}
\begin{enumerate}
	\itemsep0em
	\item Prove that $ c \ge 0$, then $\sup(cA) = c\sup A.$ 
	\item What is $\sup(cA)$ if $c \le 0$? Prove your answer is correct.
\end{enumerate}

\end{problem}

\begin{solution}
\vspace{-2mm}
\begin{enumerate}
	\itemsep0em
	\item WLOG, let $c > 0.$ Let $B_1$ be an upper bound for $A$. $\sup cA = \sup(\{ ca : a \in A\})=C_1=cB_1=c\sup A.$
	\item Prove $\sup(cA) = c\inf(A).$ Let $inf A = C_2$ and $cC_2=B_2$. So, $\{ B_2 \ge ca : a \in A\}$ since $A$ is an ordered field. $\{ ca : a\in A \}\leq B_2 \implies$ tightest upper bound is $\sup(cA).$   
\end{enumerate}
\end{solution}

\begin{problem}[Exercise 2.4]
Let $A$ be a nonempty subset of an ordered field $(F,+,\cdot,\leq)$. Assume that $\sup A$ and $\inf A$ exist in $F.$ Define $A + B := \{ a + b : a \in A, b \in B \}.$ Prove that $\sup(A + B) = \sup A + \sup B$ by filling in the details of the following outline: 
\vspace{-2mm}
\begin{itemize}
	\itemsep0em
	\item Denote $s = \sup A, t = \sup B.$ Then $s + t$ is an upper bound for $A + B.$
	\item Let $u$ be any upper bound for $A + B$, and let $a$ be any element of $A$. Then $t \leq u - a.$
	\item We have $s+t \leq u.$ Consequently, $\sup(A + B)$ exists in $F$ and is equal to $s + t = \sup A + \sup B.$ 
\end{itemize}
\end{problem}

\begin{solution}
\begin{itemize}
	\itemsep0em
	\item 
	\item
	\item
\end{itemize}
\end{solution}

\begin{problem}[Exercise 2.5.]
Let $f$ and $g$ be functions from a set $X$ to an ordered field $(F,+,\cdot,\leq).$ Let $A$ be a subset of $X.$
\end{problem}

\begin{solution}
\end{solution}

\begin{problem}[Exercise 3.3.]
Using the strategies similar to those proofs in this section, prove the following statements.
\vspace{-2mm}
\begin{enumerate}
	\itemsep0em
	\item There is no rational whose square is 20.
	\item The set $A := \{r \in \mathbb{Q} : r^2 \le 20\}$ has no least upper bound in $\mathbb{Q}$. 
\end{enumerate}
\end{problem}

\begin{solution}
\end{solution}

\begin{problem}[Exercise 4.6.]
Elements of $\mathbb{R}\setminus\mathbb{Q}$ are called \textit{irrational numbers}.
\begin{enumerate}
	\itemsep0em
	\item Assume $r$ is rational and $x$ is irrational. Show that $r + x$ and $rx$ are irrational.
	\item Use the Archimedean property of $\mathbb{R}$ to prove that the set of irrational numbers is dense in $\mathbb{R}.$ (Hint: First prove if $x$ and $y$ are real numbers with $y - x > \sqrt{2},$ then there exists an integer $m$ such that $x < m\sqrt{2} < y$.)
\end{enumerate}
\end{problem}

\begin{solution}
\begin{enumerate}
	\itemsep0em
	\item 
	\item 
\end{enumerate}
\end{solution}

\begin{problem}[Exercise 4.8.]
Assume $a, b \in \mathbb{R}.$ Prove that $a \leq b$ if and only if $a \leq b + \epsilon$ for every $\epsilon > 0.$
\end{problem}

\begin{solution}
\end{solution}

\begin{problem}[Exercise 4.9.]
Let $E$ be a set of real numbers, let $s$ be an upper bound for $E$. Prove that $s = \sup E$ if and only if for every $\epsilon > 0$ there exists $x \in E$ such that $x > s - \epsilon.$
\end{problem}

\begin{solution}
\end{solution}

\begin{problem}[Exercise 4.10.]
Let $A$ and $B$ be nonempty sets of real numbers. Decide whether the following statements are true or false. If true, give a proof; if false, give a counterexample.
\begin{enumerate}
	\itemsep0em
	\item If $\sup A < \inf B$, then there exists a $c \in \mathbb{R}$ satisfying $a < c < b$ for all $a \in A$ and $b \in B$.
	\item If there exists a $c\in \mathbb{R}$ satisfying $a < c < b$ for all $a \in A$ and $b \in B$, then $\sup A < \inf B$.
\end{enumerate}
\end{problem}

\begin{solution}
\end{solution}

\end{document}