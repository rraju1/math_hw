% HMC Math dept HW template example
% v0.04 by Eric J. Malm, 10 Mar 2005
\documentclass[12pt,letterpaper,boxed]{hmcpset}

% set 1-inch margins in the document
\usepackage[margin=1in]{geometry}

% include this if you want to import graphics files with /includegraphics
\usepackage{mathtools}
\usepackage{graphicx}
\DeclareMathOperator{\Lim}{Lim}
\DeclareMathOperator{\Int}{Int}
\DeclareMathOperator{\Img}{Im}
\DeclareMathOperator{\Cl}{Cl}
\DeclareMathOperator{\N}{\mathbb{N}}
\DeclareMathOperator{\R}{\mathbb{R}}
\DeclareMathOperator{\Q}{\mathbb{Q}}
\DeclareMathOperator{\bigep}{\mathbb{E}}
\DeclarePairedDelimiter\abs{\lvert}{\rvert}%
\DeclarePairedDelimiter{\norm}{\lVert}{\rVert}


\newcommand*\xoverline[2][0.75]{%
    \sbox{\myboxA}{$\m@th#2$}%
    \setbox\myboxB\null% Phantom box
    \ht\myboxB=\ht\myboxA%
    \dp\myboxB=\dp\myboxA%
    \wd\myboxB=#1\wd\myboxA% Scale phantom
    \sbox\myboxB{$\m@th\overline{\copy\myboxB}$}%  Overlined phantom
    \setlength\mylenA{\the\wd\myboxA}%   calc width diff
    \addtolength\mylenA{-\the\wd\myboxB}%
    \ifdim\wd\myboxB<\wd\myboxA%
       \rlap{\hskip 0.5\mylenA\usebox\myboxB}{\usebox\myboxA}%
    \else
        \hskip -0.5\mylenA\rlap{\usebox\myboxA}{\hskip 0.5\mylenA\usebox\myboxB}%
    \fi}

% info for header block in upper right hand corner
\name{Ravi Raju}
\class{MA 521}
\assignment{Homework \#5}
\duedate{2/28/2018}

\begin{document}

\problemlist{1.25, 1.26, 1.28, 1.30, 1.31, 2.7, 2.9, 2.16, 2.18.}

\begin{problem}[Exercise 1.25]
Let $(X, d)$ be a metric space. Let $E$ and $Y$ be subsets of $X$ such that $E\subset Y.$ Prove that $$\Cl_Y(E)=\Cl_X(E)\cap Y.$$
\end{problem}

\begin{solution}
For $\subset$, let $x\in \Cl_Y(E)$. $\Cl_Y(E)= E \cup \Lim_Y(E)= E \cup (\Lim_X(E)\cap Y)$ (by Exercise 1.10) $ = (E \cup \Lim_X(E))\cap Y = \Cl_X(E)\cap Y.$ So, $x \in \Cl_X(E)\cap Y.$ For $\supset$, let $x\in\Cl_X(E), Y$. So, $\Cl_X(E) = (E \cup \Lim_X(E))\cap Y = E \cup (\Lim_X(E)\cap Y) = E \cup \Lim_Y(E) = Cl_Y(E).$
\end{solution}

\begin{problem}[Exercise 1.26]
Let $(X, d)$ be a metric space.
	\vspace{-2mm}
	\begin{enumerate}
		\itemsep0em
		\item Prove that for any collection $\bigep$ of subsets of $X$, we have $$ \bigcup_{E\in\bigep} \overline{E} \subset \overline{\bigcup_{E\in\bigep} E}$$ and equality holds if $\bigep$ is finite.
		\item Prove that for any collection $\bigep$ of subsets of $X$, we have $$ \bigcap_{E\in\bigep} \overline{E} \supset \overline{\bigcap_{E\in\bigep} E}$$ and equality holds if $\bigep$ is finite.
		\item Give examples that demonstrate that equality might fail in part (1) is $\bigep$ is not finite, and equality might fail in part (2) even if $\bigep$ is finite.
	\end{enumerate}
\end{problem}

\begin{solution}
\vspace{-2mm}
	\begin{enumerate}
		\itemsep0em
		\item Let $x \in \bigcup_{E \in \bigep} \overline{E}.$ So, for some $E$, $x = E \cup \Lim_X(E).$ $\overline{\bigcup_{E\in\bigep}(E)} = [\bigcup_{E \in \bigep}E]\cup[\Lim_X(\bigcup_{E \in \bigep})].$ So, $E \subset \bigcup_{E \in \bigep}E$ and $\Lim_X (E) \subset \Lim_X(\bigcup_{E \in \bigep})$ by Exercise 1.9. So, $x \in \overline{\bigcup_{E\in\bigep} E}.$ For $\supset$, let $K = \bigcup_{E\in\bigep} \overline{E}.$ Since $K$ is union of finite number of closed sets, $K$ is a closed set. All $x\in \bigcap_{E \in \bigep}(E) \rightarrow x \in \Cl_X(E)$ and so $x \subset K.$ Thus, $\bigcap_{E \in \bigep}(E) \subset K.$ 
		\item 
		\item 
	\end{enumerate}
\end{solution}

\begin{problem}[Exercise 1.28]
Let $(X, d)$ be a metric space.
	\vspace{-2mm}
	\begin{enumerate}
		\itemsep0em
		\item Prove that $x\in X$ and $r > 0$, we have $\overline{B_X(x,r)}\subset \{y \in X : d(x,y) \leq r\}.$ (Hint: Take complements and draw a picture.) Note the inclusion $\overline{B_X(x,r)}\subset B_X(x, r + \epsilon)$ follows for any $\epsilon > 0.$
		\item Give an example using the discrete metric that demonstrates that equality need not hold in the inclusion $\overline{B_X(x,r)}\subset \{y \in X : d(x,y) \leq r\}$ that you proved in (1).
		\item Prove that in $\R^{n}$ under the Euclidean metric $d(x, y) = \norm{x - y}$, we have $\overline{B_{\R^{n}}(x,r)} = \{y \in \R^{n} : \norm{x - y} \leq r\}.$ (Again a picture might be useful)
		\item Using part (1), prove if $A$ is bounded in $(X, d)$ then $\overline{A}$ is also bounded in $(X,d)$.
	\end{enumerate}
\end{problem}

\begin{solution}
\vspace{-2mm}
	\begin{enumerate}
		\itemsep0em
		\item $\overline{B_X(x,r)} = B_X(x, r) \cup \Lim_X(B_X(x, r)).$ Let's analyze some $z \in B_X(x, r)$. For any such $z$, $d(x, z) < r \rightarrow z \in \{y \in X : d(x,y) \leq r\}$. Now assume that $y \in \Lim_X(B_X(x, r)).$ $\Lim_X(B_X(x,r)) = \{x : U \cap \{ B_X(x,r)\setminus\{x\}\neq \emptyset \}\}$ where $U$ is any neighborhood of $x$. The set of limit points will contain those points with $d(x,y) \leq r.$ Any $y$ that has $d(x, y) \ge r + \epsilon$ will violate definition of limit point. So, $\Lim_X(B_X(x,r)) \subset \{y \in X : d(x,y) \leq r\}$. This completes the inclusion, $\subset$.
		\item Let $r = 1$ and $y \in \{y \in X : d(x, y) \leq r\}. \,y \notin B_X(x, 1)=\{x\}$ and we need to show $y \notin \Lim_X(B_X(x, 1)) = U\cap (\{x\}\setminus y).$ Choose $U$ to be $B_X(y,1)$ so $y \notin \Lim_X(B_X(x,1))$.
		\item Let $D = \{y \in \R^{n} : \norm{x - y} \leq r\}$. For $\subset,$ $\overline{B_{\R^{n}}(x,r)} = B_{\R^{n}}(x,r) \cup \Lim_X(B_{\R^{n}}(x,r)).$ If $y \in B_{\R^{n}}(x,r), \, d(x, y) < r$ so $y \in D.$ If $y \in \Lim_X(B_{\R^{n}}(x,r)) = U\cap(B_{\R^{n}}(x,r)\setminus\{y\})$ for any neighborhood around $y$. This includes $y$ when $d(x, y) = r$ so $y \in D.$ For $\supset$, choose $y \in D$ s.t. $d(x,y)<r$ so $y \in B_{\R^{n}}(x,r)$. Let $z \in D$ be s.t. $d(x, z) = r$. So, for any $\epsilon > 0$, $\nexists B_X(z, \epsilon)$ s.t. $B_X(z, \epsilon)\cap(B_{\R^{n}}(x,r)\setminus\{z\}).$ Thus, $z \in \Lim_X(B_{\R^{n}}(x,r))$.
		\item Let $a \in \Lim_X(A) = B_X(x,R)\cup A \subset \{ a\}= \{\dots, b, \dots\}$. $d(a, x) \leq d(a, b) + d(b, x) \leq R.$ So, $\Lim_X(A)\in B_X(x,R)$ so $\overline{A}$ is bounded.
	\end{enumerate}
\end{solution}


\begin{problem}[Exercise 1.30]
Let $(X, d)$ be a metric space, and let $E$ be a subset of $X$. 
	\vspace{-2mm}
	\begin{enumerate}
		\itemsep0em
		\item Show that $E$ is dense in $X$ if and only if any nonempty open subset of $X$ contains a point of $E$.
		\item Suppose $E\subset Y\subset X.$ Prove that $E\cap Y$ is dense in $Y$ if and only if $\Cl_X(E) \supset Y$.
	\end{enumerate}
\end{problem}

\begin{solution}
\vspace{-2mm}
	\begin{enumerate}
		\itemsep0em
		\item For $\rightarrow$, let $x \in \Lim_X(E).$ Clearly, for any open set $U$, $U \cap E\setminus\{x\}\neq \emptyset$ so $U$ contains another point than $x$. For $\leftarrow$, let's argue via proof by contrapositive. Assume that no nonempty open subsets contain a point of $E$. So, $x\in U$ but $x \notin E$ and $x\notin \Lim_X(E)$ since $U\cap (E\setminus \{x\})=\emptyset.$ So, $x \in X$ but $x \notin \Cl_X(E).$ So, $E$ is not dense in $X$ for this case.
		\item
	\end{enumerate}
\end{solution}

\begin{problem}[Exercise 1.31]
Previously, we said that a subset $E$ of $\R$ was dense in $\R$ if for any real numbers $a$
and $b$, there exists a number $c\in E$ which lies between $a$ and $b$. Show that in $\R$, the new, more general definition of ‘dense’ agrees with the old one. That is, show that a subset $E$ of $\R$ is dense in $\R$ according
to the new definition if and only if it is dense according to the old one. (Hint: Use Exercise 1.30(1).)
\end{problem}

\begin{solution}

\end{solution}


\begin{problem}[Exercise 2.7]
Let $S = (p_n)_{n=1}^{\infty}$ be a sequence in $\R$ whose image is $(\Q\cap (0,1))\cup \{5\}.$ What are the two possibilities for $S^{*}$? Justify your answers.
\end{problem}

\begin{solution}
Set $\Img S = (\Q\cap (0,1))\cup\{5\}$. So, $(\Img S)' = [0, 1]$ and $S_{\infty}=\{ 5\}$. $S^{*}=\{[0,1]\cup \{5\}, [0,1]\}$ because 5 might appear a finite number of times in the sequence so  $5 \notin S_{\infty}.$
\end{solution}

\begin{problem}[Exercise 2.9]
Prove Proposition 2.8.
\end{problem}

\begin{solution}

\end{solution}


\begin{problem}[Exercise 2.16]
Let $(X, d)$ be a metric space, and let $(x_n)_{n=1}^{\infty}$ be a sequence in $X$. Prove the following statements.
\vspace{-2mm}
	\begin{enumerate}
		\itemsep0em
		\item If $(x_n)_{n=1}^{\infty}$ converges in $X$, then it is Cauchy in $X$.
		\item If $(x_n)_{n=1}^{\infty}$ is Cauchy in $X$, then it is bounded in $X$.
	\end{enumerate}
\end{problem}

\begin{solution}

\end{solution}

\begin{problem}[Exercise 2.18]
Let $(X, d)$ be a metric space, and let $Y$ be a subset of $X$. Prove the following statements. 
\vspace{-2mm}
	\begin{enumerate}
		\itemsep0em
		\item If $Y$ is complete, then $Y$ is closed in $X$.
		\item If $X$ is complete and $Y$ is closed in $X$, then $Y$ is complete.
	\end{enumerate}
\end{problem}

\begin{solution}

\end{solution}



\end{document}
