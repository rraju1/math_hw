% Math dept HW template example
% Ravi Raju
\documentclass[12pt,letterpaper,boxed]{hmcpset}

% set 1-inch margins in the document
\usepackage[margin=1in]{geometry}
\usepackage{mathtools}
% include this if you want to import graphics files with /includegraphics
\usepackage{graphicx}

\DeclarePairedDelimiter\abs{\lvert}{\rvert}
\DeclarePairedDelimiter{\norm}{\lVert}{\rVert}
\DeclareMathOperator{\diam}{diam}
\DeclareMathOperator{\Int}{Int}
\DeclareMathOperator{\C}{\mathbb{C}}
\DeclareMathOperator{\R}{\mathbb{R}}
\DeclareMathOperator{\N}{\mathbb{N}}
\newcommand{\iu}{{i\mkern1mu}}

\newcommand*\xoverline[2][0.75]{%
    \sbox{\myboxA}{$\m@th#2$}%
    \setbox\myboxB\null% Phantom box
    \ht\myboxB=\ht\myboxA%
    \dp\myboxB=\dp\myboxA%
    \wd\myboxB=#1\wd\myboxA% Scale phantom
    \sbox\myboxB{$\m@th\overline{\copy\myboxB}$}%  Overlined phantom
    \setlength\mylenA{\the\wd\myboxA}%   calc width diff
    \addtolength\mylenA{-\the\wd\myboxB}%
    \ifdim\wd\myboxB<\wd\myboxA%
       \rlap{\hskip 0.5\mylenA\usebox\myboxB}{\usebox\myboxA}%
    \else
        \hskip -0.5\mylenA\rlap{\usebox\myboxA}{\hskip 0.5\mylenA\usebox\myboxB}%
    \fi}

% info for header block in upper right hand corner
\name{Ravi Raju}
\class{MA 521}
\assignment{Homework \#3}
\duedate{2/14/2018}

\begin{document}

\problemlist{Chapter 2: 5.4, 5.5, 6.4 \\ Chapter 3: 1.7, 2.3, 2.4, 2.6, 2.8, 2.11, 2.12, 2.16, 2.20}

\begin{problem}[Exercise 5.4.]
Let $a$ and $b$ be real numbers. Show that the following three equalities hold: $$\bigcap_{x > b}(a,x) = (a,b], \quad \quad \bigcup_{n=1}^{\infty}[a + \frac{1}{n}, b - \frac{1}{n})=(a,b), \quad \quad \bigcap_{n = 1}^{\infty}(a + n,+\infty) = \emptyset.$$
\end{problem}


\begin{solution}
\begin{enumerate}
	\itemsep0em
	\item $\bigcap_{x > b}(a,x) = \{y \in \overline{\R} : y \in (a, x) \forall x > b\}$. $\overline{\R}$ has a least upper bound so call this bound, $d$ so $y \leq d \forall y \in \bigcap_{x > b} (a, x)$. Assume $d < b$, then $a < y \leq d < b < x \exists p \in \bigcap_{x > b}(a,x)$ s.t. $a < b < p < x.$ So, $d$ is not the least upper bound. So, $b \leq d$ and $a < y \leq b \leq d \forall \bigcap_{x>b}(a, x)$. $\bigcap_{x>b}(a, x)=\{y \in \overline{\R} : a < y \leq b\} = (a, b].$
	\item $\bigcup_{n=1}^{\infty} = [a + \frac{1}{n - 1}, b - \frac{1}{n - 1}) \cup [a + \frac{1}{n}, b - \frac{1}{n}) = \{ y \in \overline{\R} : a + \frac{1}{n} \leq y \leq b - \frac{1}{n} \} = B$.
	\vspace{-2mm}
	\begin{itemize}
		\itemsep0em
		\item $a - \frac{1}{2}$ is a lower bound since $\nexists y \in B$ s.t. $y < a + \frac{1}{n}$. Assume $\exists$ a lower bound called $\beta$ s.t. $\beta > a + \frac{1}{n}.$ If so then, $B = \{a + \frac{1}{n} < y < b - \frac{1}{n} \}.$ But, $\exists y\in B$ s.t. $y = a + \frac{1}{n}.$ So, $\nexists $ any $\beta$ so $\inf B = a + \frac{1}{n}$ and $a < a + \frac{1}{n}=\inf B.$ 
		\item $b - \frac{1}{n}$ is a upper bound since $\nexists y \in B$ s.t. $y > b - \frac{1}{n}$. Assume $\exists$ some upper bound $\alpha$ s.t. $\alpha < b - \frac{1}{n}.$ So, $b - \frac{1}{n} - \alpha > 0.$ Choose $\gamma \in \N$ so $\frac{1}{\gamma} < b - \frac{1}{n} - \alpha. (b - \frac{1}{n} - \frac{1}{\gamma}) > b - \frac{1}{n} - (b - \frac{1}{n} - \alpha).$ So, $\alpha$ is not an upper bound. So, $\sup B = b - \frac{1}{n} < b.$ 
	\end{itemize} 
	Thus, $B = \{a < \inf B \leq y < \sup B < b \forall y \in B\} = (a,b)$.
	\item 
\end{enumerate}
\end{solution}

\begin{problem}[Exercise 5.5.]
Let $a_1, a_2, \dots$ be any enumeration of the negative rational numbers; let $b_1, b_2, \dots$ be any enumeration of the positive rational numbers. Show that the following two equalities hold: $$ \bigcap_{j = 1}(a_j,b_j) = \{0\}, \quad \quad \bigcup_{j=1}^{\infty}(a_j,b_j)=\R.$$.
\end{problem}

\begin{solution}

\end{solution}



\begin{problem}[Exercise 6.4.]
Prove there exists no order $\leq$ that makes $(\mathbb{C},+,\cdot,\leq)$ into an ordered field.
\end{problem}

\begin{solution}
Let us argue by contradiction. Assume that $\mathbb{C}$ is a ordered field. Let us analyze the case $(0,1)= \iu.$ 

\vspace{-2mm}
\begin{itemize}
	\itemsep0em
	\item If $\iu > 0$, $x, y \in \C$ and $x = \iu, y = \iu$, $\iu \cdot\iu > 0 \implies -1 > 0$, which is a contradiction.
	\item If $\iu < 0$, $x, y \in \C$ and $x = -\iu, y = -\iu$ (since $x, y > 0$ for the second condition to hold), $-\iu \cdot - \iu < 0 \rightarrow \iu^{2} > 0 \implies -1 > 0$, which is a contradiction.
\end{itemize}
\end{solution}


\begin{problem}[Exercise 1.7.]
Let $\left\lVert \cdot \right\rVert$ be a norm on a real vector space $V$. Prove the \textit{reverse triangle inequality}: $$\abs{\left\lVert x \right\rVert - \left\lVert y \right\rVert} \leq \left\lVert x - y \right\rVert$$
\end{problem}

\begin{solution}
$\norm{x}= \norm{(x - y) + y} \leq \abs{\norm{x - y} + \norm{y}}$. $ \abs{\norm{x} - \norm{y}} \leq \abs{\norm{(x - y) + \norm{y} - \norm{y}}} = \abs{\norm{(x - y)}}=\norm{x - y}.$ \\
$\norm{y}= \norm{(y - x) + x}=\norm{y - x} + \norm{x}. $$\abs{\norm{x} - \norm{y}} \leq \abs{\norm{x} - (\norm{y - x} + \norm{x})} \leq \norm{x - y}.$

\end{solution}

\begin{problem}[Exercise 2.3.]
Let $X$ be any set. Prove that the discrete metric $d : X \times X \rightarrow \mathbb{R}$(defined by $d(x,y)=1$ if $x \neq y$ and $d(x, x) = 0$ for $x \in X$) satisfies the triangle inequality and is therefore a metric on $X$.
\end{problem}

\begin{solution}
Let's argue via proof by contrapositive. Assume $d(x, y) > d(x, z) + d(z, y) \, \forall x, y, z \in X$ so $d$ is not a metric on $X.$ Let $x \neq y$ and enumerate what values $z$ can take on. 
\vspace{-2mm}
\begin{itemize}
	\itemsep0em
	\item If $z \neq y \neq$ z, then $1 > 2.$
	\item If $z = x$ and $z \neq y$, vice versa, then $1 > 1.$
	\item If $z = y = x,$ then $1 > 0.$ However, this contradicts our initial assumption that $x \neq y.$
\end{itemize}
Clearly, all these pose contradictions, so it must be that $d$ satisfies the triangle inequality and is a metric on $X$.

\end{solution}

\begin{problem}[Exercise 2.4.]
Determine which of the following functions are metrics on $\mathbb{R}$. Prove your answer in each case.
\vspace{-2mm}
\begin{itemize}
	\itemsep0em
	\item $d_1(x,y)=\sqrt{\abs{x - y}}.$
	\item $d_2(x,y)=\abs{x - 2y}.$
	\item $d_3(x,y)=\frac{\abs{x - y}}{1 + \abs{x - y}}.$
\end{itemize}
\end{problem}
\begin{solution}

\end{solution}

\begin{problem}[Exercise 2.6.]
Let $\norm{\cdot}$ denote the Euclidean norm $\mathbb{R}^{2},$ i.e. $\norm{(x_1, x_2)} = \sqrt{x_1^{2} + x_2^{2}}.$ Consider the function from $\mathbb{R}^{2} \times \mathbb{R}^{2} \rightarrow \mathbb{R}$, defined by $$d(x, y) = \norm{x_1 - y_1} + \norm{x_2 - y_2}, \quad \quad (x = (x_1, x_2), y = (y_1, y_2)). $$
\vspace{-2mm}
\begin{enumerate}
	\itemsep0em
	\item Prove that $d$ is a metric on $\mathbb{R}^{2}.$
	\item On a sheet of graph paper, draw the set $B_{d}((5, 1), 3).$ Use dotted lines to indicate the ‘boundary’,
which is not included in the set you are drawing. (Hint: it may be easier to figure out what the
set looks like if you first consider $B_{d}((0, 0), 3).$
	\item On the same graph as in the previous part, draw $B_{d_{u}}((-3, 2),1)$, where $d_{u}$ denotes the square metric.
\end{enumerate}
\end{problem}
\begin{solution}

\end{solution}

\begin{problem}[Exercise 2.8.]
Let ($X, d$) be a metric space, and let $E$ be a subset of $X$. The \textit{diameter} of $E$ in ($X,d$) is defined by the formula $$\diam_{d}(E) = \sup\{d(x,y) : x,y \in E\}.$$

\vspace{-2mm}
\begin{enumerate}
	\itemsep0em
	\item Prove that for any $r > 0$ and $x \in X$, we have $\diam(B(x,r))\leq 2r.$
	\item If $X$ is any set and $d$ is the discrete metric, show $\diam(B(x, r)) = 0.$
	\item If $X = \mathbb{R}^{n}$ for some $n \in \mathbb{N}$ and $d$ is the Euclidean metric, prove that $\diam(B(x,r)) = 2r.$ 
\end{enumerate}

\end{problem}
\begin{solution}

\end{solution}

\begin{problem}[Exercise 2.11.]
As in Example 2.7, let $X = \mathbb{R}^{2}, Y = [-1,3]\times[2,4],$ and let $d$ denote the Euclidean metric on $X = \mathbb{R}^{2}.$ Let $p = (3,4)$ and let $q = (2, 4)$. Arguing \textit{directly from the definition of an interior point} (i.e., without using Exercise 2.12), show that $q$ is an interior point $B_{Y}(p,2)$ with respect to $Y$, but $q$ is not an interior point $B_{Y}(p,2)$ with respect to $X$. In addition, draw a picture on a piece of graph paper that illustrates the idea of your proof.
\end{problem}
\begin{solution}

\end{solution}


\begin{problem}[Exercise 2.12.]
Let $(X, d)$ be a metric space, and let $Y$ be a subset of $X$. Prove that $$(*) \Int_{X}(U) = \Int_{Y}(U)\cap\Int_{X}(Y).$$
In the notation of Exercise 2.11, the equality $(*)$ gives an alternate explanation of why q is not an interior point of $B_{Y}(p, 2)$ with respect to $X$: It is because $q \notin \Int_{X}(Y)$, as can be seen from the picture you drew in that Exercise.
\end{problem}
\begin{solution}

\end{solution}

\begin{problem}[Exercise 2.16.]
Let $(X, d)$ be a metric space, and let $U$ be a subset of $X$. Use Proposition 2.15 to prove that $\Int_{X}(U)$ is open in $X$.
\end{problem}
\begin{solution}

\end{solution}

\begin{problem}[Exercise 2.20.]
Let $(X, d)$ be a metric space. Assume that $U\subset Y\subset X,$ and additionally that $Y$ is open $X$. Prove that $U$ is open in $Y$ if and only if $U$ is open in $X$. (Note: There at least two possible solutions; one uses Theorem 2.19, the other uses Exercise 2.12.)
\end{problem}
\begin{solution}

\end{solution}
\end{document}