% Math dept HW template example
% Ravi Raju
\documentclass[12pt,letterpaper,boxed]{hmcpset}

% set 1-inch margins in the document
\usepackage[margin=1in]{geometry}
\usepackage{mathtools}
% include this if you want to import graphics files with /includegraphics
\usepackage{graphicx}

\DeclarePairedDelimiter\abs{\lvert}{\rvert}
\DeclarePairedDelimiter{\norm}{\lVert}{\rVert}
\DeclareMathOperator{\diam}{diam}
\DeclareMathOperator{\Int}{Int}
\DeclareMathOperator{\C}{\mathbb{C}}
\DeclareMathOperator{\R}{\mathbb{R}}
\DeclareMathOperator{\Q}{\mathbb{Q}}
\DeclareMathOperator{\N}{\mathbb{N}}
\newcommand{\iu}{{i\mkern1mu}}

\newcommand*\xoverline[2][0.75]{%
    \sbox{\myboxA}{$\m@th#2$}%
    \setbox\myboxB\null% Phantom box
    \ht\myboxB=\ht\myboxA%
    \dp\myboxB=\dp\myboxA%
    \wd\myboxB=#1\wd\myboxA% Scale phantom
    \sbox\myboxB{$\m@th\overline{\copy\myboxB}$}%  Overlined phantom
    \setlength\mylenA{\the\wd\myboxA}%   calc width diff
    \addtolength\mylenA{-\the\wd\myboxB}%
    \ifdim\wd\myboxB<\wd\myboxA%
       \rlap{\hskip 0.5\mylenA\usebox\myboxB}{\usebox\myboxA}%
    \else
        \hskip -0.5\mylenA\rlap{\usebox\myboxA}{\hskip 0.5\mylenA\usebox\myboxB}%
    \fi}

% info for header block in upper right hand corner
\name{Ravi Raju}
\class{MA 521}
\assignment{Homework \#3}
\duedate{2/14/2018}

\begin{document}

\problemlist{Chapter 2: 5.4, 5.5, 6.4 \\ Chapter 3: 1.7, 2.3, 2.4, 2.6, 2.8, 2.11, 2.12, 2.16, 2.20}

\begin{problem}[Exercise 5.4.]
Let $a$ and $b$ be real numbers. Show that the following three equalities hold: $$\bigcap_{x > b}(a,x) = (a,b], \quad \quad \bigcup_{n=1}^{\infty}[a + \frac{1}{n}, b - \frac{1}{n})=(a,b), \quad \quad \bigcap_{n = 1}^{\infty}(a + n,+\infty) = \emptyset.$$
\end{problem}


\begin{solution}
\begin{enumerate}
	\itemsep0em
	\item $\bigcap_{x > b}(a,x) = \{y \in \overline{\R} : y \in (a, x) \forall x > b\}$. Clearly, $(a,b]\subset \bigcap_{x > b}(a,x)$ since it is contained in every set of the general intersection. For the other inclusion, let $z$ be an upper bound of $\bigcap_{x > b}(a,x)$ where $z < x \, \forall x > b$. For any $\epsilon > 0$, $z = b + \epsilon$. By Exer. 4.8, $b \leq z \rightarrow b \leq b.$ So, $b$ is upper bound on $\bigcap_{x > b}(a,x)$ so $\bigcap_{x > b}(a,x)\subset (a,b].$
	\item $\bigcup_{n=1}^{\infty} = [a + \frac{1}{n - 1}, b - \frac{1}{n - 1}) \cup [a + \frac{1}{n}, b - \frac{1}{n}) = \{ y \in \overline{\R} : a + \frac{1}{n} \leq y \leq b - \frac{1}{n} \} = B$.
	\vspace{-2mm}
	\begin{itemize}
		\itemsep0em
		\item $a - \frac{1}{2}$ is a lower bound since $\nexists y \in B$ s.t. $y < a + \frac{1}{n}$. Assume $\exists$ a lower bound called $\beta$ s.t. $\beta > a + \frac{1}{n}.$ If so then, $B = \{a + \frac{1}{n} < y < b - \frac{1}{n} \}.$ But, $\exists y\in B$ s.t. $y = a + \frac{1}{n}.$ So, $\nexists $ any $\beta$ so $\inf B = a + \frac{1}{n}$ and $a < a + \frac{1}{n}=\inf B.$ 
		\item $b - \frac{1}{n}$ is a upper bound since $\nexists y \in B$ s.t. $y > b - \frac{1}{n}$. Assume $\exists$ some upper bound $\alpha$ s.t. $\alpha < b - \frac{1}{n}.$ So, $b - \frac{1}{n} - \alpha > 0.$ Choose $\gamma \in \N$ so $\frac{1}{\gamma} < b - \frac{1}{n} - \alpha. (b - \frac{1}{n} - \frac{1}{\gamma}) > b - \frac{1}{n} - (b - \frac{1}{n} - \alpha).$ So, $\alpha$ is not an upper bound. So, $\sup B = b - \frac{1}{n} < b.$ 
	\end{itemize} 
	Thus, $B = \{a < \inf B \leq y < \sup B < b \forall y \in B\} = (a,b)$.
	\item Let's do proof by contradiction. Assume $\bigcap_{n = 1}^{\infty}(a + n,+\infty) = X$ s.t. $X = \{\beta\}, \beta \in \R, \beta < a + n.$ Enumerate $\bigcap_{n = 1}^{\infty}(a + n,+\infty) = (a + 1, \infty)\cap(a + 2, \infty)\cap\dots \cap(a + n, \infty)\cap \dots$. Take a look at set, $(a + n, \infty) = \{x \in \overline{\R} : a+n < x < \infty \} = B.$ Clearly $a + n$ is a lower bound for $B$ since $a+n<x \, \forall x\in B.$ Assume that $k$ is a lower bound s.t. $k > a + n.$ So, $B = \{ x \in \overline{\R} : a + n < k < x < \infty \}$. So, $k - a - n > 0$. Choose $\phi$ so $\phi > k - a - n. a + n + \phi > a + n + k - a - n.$ So, $k$ is not a lower bound. So, $\inf B = a + n \rightarrow \beta < a+ n=\inf B. \beta \notin B$ so $\beta \notin \bigcap_{n = 1}^{\infty}(a + n,+\infty)$. This establishes a contradiction so $\bigcap_{n = 1}^{\infty}(a + n,+\infty) = \emptyset.$
\end{enumerate}
\end{solution}

\begin{problem}[Exercise 5.5.]
Let $a_1, a_2, \dots$ be any enumeration of the negative rational numbers; let $b_1, b_2, \dots$ be any enumeration of the positive rational numbers. Show that the following two equalities hold: $$ \bigcap_{j = 1}^{\infty}(a_j,b_j) = \{0\}, \quad \quad \bigcup_{j=1}^{\infty}(a_j,b_j)=\R.$$.
\end{problem}

\begin{solution}
\begin{enumerate}
	\item Let's analyze one set of this general intersection, $(a_n, b_n)$ and call it $A$. $\sup A$ exists by the least upper bound property since $A\subset\R$ is bounded above and nonempty by least upper bound property. Similar argument for existence of $\inf A$ but it is bounded below. Let us prove what $\sup A, \inf A$ are supposed to be.
	\vspace{-2mm}
	\begin{itemize}
		\itemsep0em
		\item $a_n$ is a lower bound $\rightarrow x \notin A$ s.t. $x < a_n \, \forall x \in A.$ Assume that $l$ is a lower bound s.t. $l > a_n.$ So, $A = \{ x \in \overline{\R} : a_n < k < x < b_n \}$. So, $l - a_n > 0$. Choose $\alpha \in \N$ so $\alpha > l - a_n.\, \alpha + a_n > a_n + l - a_n.$ Thus, $l$ is not a lower bound. So, $\inf A = a_n.$
		\item $b_n$ is a upper bound $\rightarrow x \notin A$ s.t. $x > b_n \, \forall x \in A.$ Assume that $u$ is an upper bound s.t. $u < b_n.$ So, $A = \{ x \in \overline{\R} : a_n < x < u < b_n \}$. So, $u -  b_n > 0$. Choose $n \in \N$ so $u - b_n > \frac{1}{n}.$ So, $b_n - \frac{1}{n} > u$. Thus, $u$ is not an upper bound. So, $\sup A = b_n.$
	\end{itemize}
	Since $a_n \in - \Q$ and $b_n \in +\Q \, \forall n \in \N, 0 \in (a_n, b_n) \, \forall n$ so $\bigcap_{j=1}^{\infty} (a_n,b_n)= \{0\}.$

	\item Let's analyze one set of this general union, $(a_n, b_n)$ and call it $A$. $\sup A$ exists by the least upper bound property since $A\subset\R$ is bounded above and nonempty by least upper bound property. Similar argument for existence of $\inf A$ but it is bounded below. Let us prove what $\sup A, \inf A$ are supposed to be.
	\vspace{-2mm}
	\begin{itemize}
		\itemsep0em
		\item $a_n$ is a lower bound $\rightarrow x \notin A$ s.t. $x < a_n \, \forall x \in A.$ Assume that $l$ is a lower bound s.t. $l > a_n.$ So, $A = \{ x \in \overline{\R} : a_n < k < x < b_n \}$. So, $l - a_n > 0$. Choose $\alpha \in \N$ so $\alpha > l - a_n.\, \alpha + a_n > a_n + l - a_n.$ Thus, $l$ is not a lower bound. So, $\inf A = a_n.$
		\item $b_n$ is a upper bound $\rightarrow x \notin A$ s.t. $x > b_n \, \forall x \in A.$ Assume that $u$ is an upper bound s.t. $u < b_n.$ So, $A = \{ x \in \overline{\R} : a_n < x < u < b_n \}$. So, $u -  b_n > 0$. Choose $n \in \N$ so $u - b_n > \frac{1}{n}.$ So, $b_n - \frac{1}{n} > u$. Thus, $u$ is not an upper bound. So, $\sup A = b_n.$
	\end{itemize}
	Assume that there is some $\beta \in \R$. If $\beta \notin A$, $\exists$ some set in  $\bigcup_{j=1}^{\infty}(a_j,b_j)$ s.t. $\beta \in \{ a_n - \epsilon < x < b_n + \epsilon\}$ for any $\epsilon > 0.$ So, if true for any $\beta$, $\bigcup_{j=1}^{\infty}(a_j,b_j)= \R.$
\end{enumerate}
\end{solution}

\begin{problem}[Exercise 6.4.]
Prove there exists no order $\leq$ that makes $(\mathbb{C},+,\cdot,\leq)$ into an ordered field.
\end{problem}

\begin{solution}
Let us argue by contradiction. Assume that $\mathbb{C}$ is a ordered field. Let us analyze the case $(0,1)= \iu.$ 

\vspace{-2mm}
\begin{itemize}
	\itemsep0em
	\item If $\iu > 0$, $x, y \in \C$ and $x = \iu, y = \iu$, $\iu \cdot\iu > 0 \implies -1 > 0$, which is a contradiction.
	\item If $\iu < 0$, $x, y \in \C$ and $x = -\iu, y = -\iu$ (since $x, y > 0$ for the second condition to hold), $-\iu \cdot - \iu < 0 \rightarrow \iu^{2} > 0 \implies -1 > 0$, which is a contradiction.
\end{itemize}
\end{solution}


\begin{problem}[Exercise 1.7.]
Let $\left\lVert \cdot \right\rVert$ be a norm on a real vector space $V$. Prove the \textit{reverse triangle inequality}: $$\abs{\left\lVert x \right\rVert - \left\lVert y \right\rVert} \leq \left\lVert x - y \right\rVert$$
\end{problem}

\begin{solution}
$\norm{x}= \norm{(x - y) + y} \leq \abs{\norm{x - y} + \norm{y}}$. $ \abs{\norm{x} - \norm{y}} \leq \abs{\norm{(x - y) + \norm{y} - \norm{y}}} = \abs{\norm{(x - y)}}=\norm{x - y}.$ \\
$\norm{y}= \norm{(y - x) + x}=\norm{y - x} + \norm{x}. $$\abs{\norm{x} - \norm{y}} \leq \abs{\norm{x} - (\norm{y - x} + \norm{x})} \leq \norm{x - y}.$

\end{solution}

\begin{problem}[Exercise 2.3.]
Let $X$ be any set. Prove that the discrete metric $d : X \times X \rightarrow \mathbb{R}$(defined by $d(x,y)=1$ if $x \neq y$ and $d(x, x) = 0$ for $x \in X$) satisfies the triangle inequality and is therefore a metric on $X$.
\end{problem}

\begin{solution}
Let's argue via proof by contrapositive. Assume $d(x, y) > d(x, z) + d(z, y) \, \forall x, y, z \in X$ so $d$ is not a metric on $X.$ Let $x \neq y$ and enumerate what values $z$ can take on. 
\vspace{-2mm}
\begin{itemize}
	\itemsep0em
	\item If $z \neq y \neq$ z, then $1 > 2.$
	\item If $z = x$ and $z \neq y$, vice versa, then $1 > 1.$
	\item If $z = y = x,$ then $1 > 0.$ However, this contradicts our initial assumption that $x \neq y.$
\end{itemize}
Clearly, all these pose contradictions, so it must be that $d$ satisfies the triangle inequality and is a metric on $X$.

\end{solution}

\begin{problem}[Exercise 2.4.]
Determine which of the following functions are metrics on $\mathbb{R}$. Prove your answer in each case.
\vspace{-2mm}
\begin{itemize}
	\itemsep0em
	\item $d_1(x,y)=\sqrt{\abs{x - y}}.$
	\item $d_2(x,y)=\abs{x - 2y}.$
	\item $d_3(x,y)=\frac{\abs{x - y}}{1 + \abs{x - y}}.$
\end{itemize}
\end{problem}
\begin{solution}
\vspace{-2mm}
\begin{enumerate}
	\itemsep0em
	\item For any $x,y \in \R$, $\abs{x - y}\ge 0$ so $\sqrt{\abs{x - y}}\ge 0$. $d_1(x,y)= \sqrt{\abs{x - y}}= \sqrt{\abs{(-1)y -x}} = \sqrt{\abs{y -x}}=d_1(y,x).$ $\sqrt{\abs{x - y}} \leq \sqrt{\abs{x - z}} + \sqrt{\abs{z - y}} \rightarrow \abs{x - y} \leq \abs{x-z} + c + \abs{z - y}$ where $c \leq 0$. So, $-(\abs{x-z} + c + \abs{z - y}) \leq x - y \leq \abs{x-z} + c + \abs{z - y} \rightarrow -c \leq 0 \leq c$. So $d_1$ is a metric on $\R.$ 
	\item Assume $d_2(x,y)=d_2(y,x)\rightarrow \abs{x - 2y} = \abs{y - 2x} \forall \, x, y \in \R.$ Choose $x=0$ and $y=1$ so $d_2(x,y)=1=d_2(y,x)=2$. So, $d_2$ is clearly not a metric on $\R.$
	\item For any $x,y \in \R$, $\abs{x - y}\ge 0$ so $\frac{\abs{x - y}}{1 + \abs{x - y}} \ge 0.$ $d_3(x,y)= \frac{\abs{x - y}}{1 + \abs{x - y}}= \frac{\abs{(-1)y -x}}{1 + \abs{(-1)y -x}} = \frac{\abs{y -x}}{1 + \abs{y -x}} =d_3(y,x)$. Assume $\frac{\abs{x - y}}{1 + \abs{x - y}} > \frac{\abs{x - y}}{1 + \abs{x - y}} + \frac{\abs{x - y}}{1 + \abs{x - y}}.$ Choose $x, y = 5$ and $z = 0$. So, $0 > \frac{\abs{5}}{6} + \frac{\abs{5}}{6}$ and clearly this is a contradiction so $d_3$ follows triangle inequality. So, $d_3$ is a metric on $\R$.
\end{enumerate}
\end{solution}

\begin{problem}[Exercise 2.6.]
Consider the function $d: \R^{2} \times \R^{2} \rightarrow \R,$ defined by $$d(x, y) = \abs{x_1 - y_1} + \abs{x_2 - y_2}, \quad \quad (x = (x_1, x_2), y = (y_1, y_2)). $$
\vspace{-2mm}
\begin{enumerate}
	\itemsep0em
	\item Prove that $d$ is a metric on $\mathbb{R}^{2}.$
	\item On a sheet of graph paper, draw the set $B_{d}((5, 1), 3).$ Use dotted lines to indicate the ‘boundary’,
which is not included in the set you are drawing. (Hint: it may be easier to figure out what the
set looks like if you first consider $B_{d}((0, 0), 3).$)
	\item On the same graph as in the previous part, draw $B_{d_{u}}((-3, 2),1)$, where $d_{u}$ denotes the square metric.
\end{enumerate}
\end{problem}
\begin{solution}
For nonnegativity, analyze two cases where $x, y \in \R^{2}$.
\begin{enumerate}
	\itemsep0em
	\item Case 1: Let  $(x_1, x_2) = (y_1, y_2)$. So, $d(x, y) = \abs{x_1 - y_1} + \abs{x_2 - y_2} = \abs{x_1 - x_1} + \abs{x_2 - x_2} = 0.$ 
	\item Case 2: Let $x \neq y \, \forall x, y \in \R^{2}.$ Claim that $d(x, y) < 0$ so $d$ is not a metric. $d(x, y) = \abs{x_1 - y_1} + \abs{x_2 - y_2} < 0.$ $\abs{x_1 - y_1} < -\abs{x_2 - y_2}$. Choose $x = (0, 0)$ and $y = (1, 2)$. So, $\abs{0 - 1} < -\abs{0 - 2} \rightarrow 1 < -2.$ Clearly, this is false so $d(x,y) \ge 0.$ 
\end{enumerate}

For symmetry, $d(x, y) = \abs{x_1 - y_1} + \abs{x_2 - y_2} = \abs{(-1)y_1 - x_1} + \abs{(-1)y_2 - x_2} = \abs{y_1 - x_1} + \abs{y_2 - x_2} = d(y,x).$ \\
For triangle inequality, $d(x,y) = \abs{x_1 - y_1} + \abs{x_2 - y_2} \leq \abs{x_1 - z_1} + \abs{z_1 - y_1} + \abs{x_2 - z_2} + \abs{z_2 - y_2} = d(x, z) + d(z, y).$ 
Drawings attached on back
\end{solution}

\begin{problem}[Exercise 2.8.]
Let ($X, d$) be a metric space, and let $E$ be a subset of $X$. The \textit{diameter} of $E$ in ($X,d$) is defined by the formula $$\diam_{d}(E) = \sup\{d(x,y) : x,y \in E\}.$$

\vspace{-2mm}
\begin{enumerate}
	\itemsep0em
	\item Prove that for any $r > 0$ and $x \in X$, we have $\diam(B(x,r))\leq 2r.$
	\item If $X$ is any set and $d$ is the discrete metric, show $\diam(B(x, r)) = 0$ for any $r \leq 1,$ while $\diam(B(x,r)) = 1$ for any $r > 1$.
	\item If $X = \mathbb{R}^{n}$ for some $n \in \mathbb{N}$ and $d$ is the Euclidean metric, prove that $\diam(B(x,r)) = 2r.$ 
\end{enumerate}

\end{problem}
\begin{solution}

\vspace{-2mm}
\begin{enumerate}
	\itemsep0em
	\item $B(x, r)= \{ y \in X : d(x, y) < r\}.$ $\diam_d(\{ y \in X : d(x, y) : x, y \in B(x,r)\}) = \sup\{ d(x,y) : x, y \in B(x,r)\} \leq \sup\{ d(x,z) : z, y \in B(x,r)\} + \sup\{ d(z,y) : y, z \in B(x,r)\}.$ So, $\diam_d(\{y \in X : d(x, y) : x, y \in B(x,r)\})\leq r + r = 2r.$
	\item Recall discrete metric is defined as $d(x, y) = 1$ if $x \neq y$ and $d(x, y) = 0$ if $x = y$. 
	\vspace{-2mm}
	\begin{enumerate}
		\itemsep0em
		\item For $r \leq 1,$ $B_{(X, d)}(x, r) = \{ y \in X : d(x, y) < r \}$. The cases where $x \neq y$ will yield an empty set since $1$ is not greater $1$. So, we will only have the set,$D$, where $x = y.$ So, $\sup\{d(x,y) : x, y \in D\} = 0.$
		\item For $r > 1,$ $B_{(X, d)}(x, r) = \{ y \in X : d(x, y) < r \} = \{y \in X : x \neq y\}\cap\{y \in X : x \in y\} = B.$ $\sup\{d(x,y) : x, y \in B \} = 1$. Any point, $y$ in the metric space will fulfill the condition from the ball since $d$ is the discrete metric. The maximum can only be $1$ so it must be the $\sup$ of the set.
	\end{enumerate}

	\item Choose two points $a_1, a_2 \in B(x,r)$ s.t. $a_1 = [x_1 + r - \epsilon, x_2, x_3, \dots], a_2 = [x_1 - (r - \epsilon), x_2, x_3, \dots].$ $\diam{B(x,r)}= \sup\{d(a_1, a_2): a_1, a_2 \in B(x,r)\}.$ Apply triangle inequality to $d(a_1, a_2) \rightarrow d(a_1, a_2) \leq d(a_1, x) + d(x, a_2).$ So, $d(a_1, a_2) \leq \norm{r - \epsilon} + \norm{(-1)(r - \epsilon)}.$ $d(a_1,a_2)\leq 2r -2\epsilon \forall \epsilon > 0.$ Use Exer. 4.8, $d(a_1,a_2) + 2\epsilon \leq 2r \rightarrow d(a_1,a_2) \leq 2r.$ So, $\diam{B(x,r)}= \sup\{d(a_1,a_2)\leq 2r\}=2r.$
\end{enumerate}
\end{solution}

\begin{problem}[Exercise 2.11.]
As in Example 2.7, let $X = \mathbb{R}^{2}, Y = [-1,3]\times[2,4],$ and let $d$ denote the Euclidean metric on $X = \mathbb{R}^{2}.$ Let $p = (3,4)$ and let $q = (2, 4)$. Arguing \textit{directly from the definition of an interior point} (i.e., without using Exercise 2.12), show that $q$ is an interior point of $B_{Y}(p,2)$ with respect to $Y$, but $q$ is not an interior point of $B_{Y}(p,2)$ with respect to $X$. In addition, draw a picture on a piece of graph paper that illustrates the idea of your proof.
\end{problem}
\begin{solution}
Let $D=B_Y(p,2)$ with respect to $Y$. To show that $q$ is an interior point of $D$, $\exists r > 0$ s.t. $B_Y(q,r)\subset D$. Set $r = 1$ so for some $x \in B_Y(q,1)$ s.t. $d(x,q) < 1$ and $x \in Y.$ By triangle inequality, $d(x,p) \leq d(x, q) + d(p, q)$. $d(x, q) < 1$ and $d(p, q) = 1$ so $d(x, q) + d(p, q) < 2 \rightarrow x \in D.$ So, $B_Y(q,1) \subset D \rightarrow q \in \Int_Y(D).$ \\
Let $E=B_Y(p,2)$. To show that $q$ is not an interior point of $E$, $x \in B_X(q, r), x \notin E$, for some $r > 0$. Take $x = (2, 4 + \frac{r}{2}). \, d(x, q) = \frac{r}{2} \rightarrow x\in B_X(q, r).$ But, $x \notin Y$ since $(4 + \frac{r}{2} > 4).$ So, $q$ is not an interior point of $E$ with respect to $X.$\\
Drawings attached on back
\end{solution}


\begin{problem}[Exercise 2.12.]
Let $(X, d)$ be a metric space, and let $Y$ be a subset of $X$. Prove that for any subset $U$ of $Y$, we have $$(*) \Int_{X}(U) = \Int_{Y}(U)\cap\Int_{X}(Y).$$
In the notation of Exercise 2.11, the equality $(*)$ gives an alternate explanation of why q is not an interior point of $B_{Y}(p, 2)$ with respect to $X$: It is because $q \notin \Int_{X}(Y)$, as can be seen from the picture you drew in that Exercise.
\end{problem}
\begin{solution}
$\Int_{X}(U)=\{ x \in U : x$ is an interior point of $U$ with respect to $X \}$. That means there is some $B_X(x, r) \subset U,$ s.t. $r > 0.$ Call this ball, $D$. Since $D\subset U \rightarrow D \subset \Int_{Y}(U).$ Also, since $U\subset Y,$ $D\subset Y.$ So, $D \subset \Int_{X}(Y)$. So, $\Int_{X}(U) \subset \Int_{Y}(U)\cap\Int_{X}(Y).$\\
Now for the other direction, $\Int_{Y}(U)=\{ x \in U : x$ is an interior point of $U $ with respect to $Y \}.$ That means there is some $B_Y(x, r) \subset U,$ s.t. $r > 0.$ Call this ball, $E$. Since $E\subset U$, $E \subset Y \rightarrow E \subset \Int_{X}(Y).$ So, $E \subset \Int_{Y}(U)\cap\Int_{X}(Y)\subset U \subset \Int_{X}(U).$ So, $\Int_{X}(U) = \Int_{Y}(U)\cap\Int_{X}(Y)$.
\end{solution}

\begin{problem}[Exercise 2.16.]
Let $(X, d)$ be a metric space, and let $U$ be a subset of $X$. Use Proposition 2.15 to prove that $\Int_{X}(U)$ is open in $X$.
\end{problem}
\begin{solution}
For $x \in U$, since $U\subset X$ and $(X,d)$ is a metric space, $B_X(x, r)$ is open in $X$. Choose $y\in U$ so $y$ is the interior point of $B_X(y,r)$ using Proposition 2.15. $B_X(y, r)\cap U = B_U(y, r) \subset U.$ So, $y$ is an interior point of $U$ wrt $X$ so $\forall y \in U$ are interior points so $\Int_X U = U.$
\end{solution}

\begin{problem}[Exercise 2.20.]
Let $(X, d)$ be a metric space. Assume that $U\subset Y\subset X,$ and additionally that $Y$ is open $X$. Prove that $U$ is open in $Y$ if and only if $U$ is open in $X$. (Note: There at least two possible solutions; one uses Theorem 2.19, the other uses Exercise 2.12.)
\end{problem}
\begin{solution}

\end{solution}
\end{document}