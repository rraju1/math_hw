% HMC Math dept HW template example
% v0.04 by Eric J. Malm, 10 Mar 2005
\documentclass[12pt,letterpaper,boxed]{hmcpset}

% set 1-inch margins in the document
\usepackage[margin=1in]{geometry}

% include this if you want to import graphics files with /includegraphics
\usepackage{mathtools}
\usepackage{graphicx}
\DeclareMathOperator{\Lim}{Lim}
\DeclareMathOperator{\Int}{Int}
\DeclareMathOperator{\Img}{Im}
\DeclareMathOperator{\Cl}{Cl}
\DeclareMathOperator{\A}{\mathcal{A}}
\DeclareMathOperator{\N}{\mathbb{N}}
\DeclareMathOperator{\R}{\mathbb{R}}
\DeclareMathOperator{\Q}{\mathbb{Q}}
\DeclareMathOperator{\bigep}{\mathcal{E}}
\DeclarePairedDelimiter\abs{\lvert}{\rvert}%
\DeclarePairedDelimiter{\norm}{\lVert}{\rVert}

\newcommand*\xoverline[2][0.75]{%
    \sbox{\myboxA}{$\m@th#2$}%
    \setbox\myboxB\null% Phantom box
    \ht\myboxB=\ht\myboxA%
    \dp\myboxB=\dp\myboxA%
    \wd\myboxB=#1\wd\myboxA% Scale phantom
    \sbox\myboxB{$\m@th\overline{\copy\myboxB}$}%  Overlined phantom
    \setlength\mylenA{\the\wd\myboxA}%   calc width diff
    \addtolength\mylenA{-\the\wd\myboxB}%
    \ifdim\wd\myboxB<\wd\myboxA%
       \rlap{\hskip 0.5\mylenA\usebox\myboxB}{\usebox\myboxA}%
    \else
        \hskip -0.5\mylenA\rlap{\usebox\myboxA}{\hskip 0.5\mylenA\usebox\myboxB}%
    \fi}

% info for header block in upper right hand corner
\name{Ravi Raju}
\class{MA 521}
\assignment{Homework \#11}
\duedate{5/2/2019}

\begin{document}

\problemlist{Chapter 8: 2.12, 2.15, 2.16, 2.17,\\
             2.18, 2.19, 2.20, 3.3}

\begin{problem}[Exercise 2.12]
Which $n\in\N$ have the property that $f^{n} \in \mathcal{R}([a,b])$ implies $f\in\mathcal{R}([a,b])$? Give proofs(s) and counterexamples(s) to show your answer is correct and complete.
\end{problem}

\begin{solution}
We knwo that if $f$ is differentiable on $[a,b]$ that is must be continuous on $[a,b]$. By contradition, we will show that $f$ is bounded. If $f$ is unbounded at $[a,b]$ then $\exists x\in[a,b]$ s.t. $\nexists M$ s.t. $\abs{x} < M$. Let's analyze the limit as $\lim_{h \rightarrow 0}\frac{f(x + h) - f(x)}{h} = f'(x)$. Without loss of generality, assume $f(x + h)$ is finite. By our previous statement, we claimed that $f(x)$ was unbounded which implies this limit can't exist. This means that $f^{1} \notin \mathcal{R}([a,b])$ which is a contradition. Since $f$ is continuous and bounded, we can say that by Theorem 2.8 $f\in\mathcal{R}([a,b]).$ If this holds for $f'$, then it will hold for all $n > 1$ as well.
\end{solution}

\begin{problem}[Exercise 2.15]
Show that if $f : [a, b] \rightarrow \R$ is continuous and $F(x) := \int_{a}^{x} f(t) dt = 0$ for all $x \in [a, b]$, then $f(x)=0$ for all $x \in [a,b].$ Provide an example to show that the statement is false if $f$ is not continuous.
\end{problem}

\begin{solution}
We want $\lim_{t\rightarrow x} \frac{F(t) - F(x)}{t - x} = F'(x) = f(x) = 0\, \forall x \in [a, b]$. We know that $F(x) := \int_{a}^{x} f(t) dt = 0.$ So this means that $\forall x\in[a,b], F(x)=F(a)$ by the Fundamental Theorem of Calculus. So then, substitue $F(a)$ for $F(x)$ and $\lim_{t \rightarrow x} \frac{F(t) - F(a)}{t - x} = F'(x)= f(x) = 0 \forall x\in[a, b].$ Let's pick the same function $f(x) = 0$ but at some point $x \in [a,b]$, there exists a removable discontinuity where $f(x) = 1$. If we choose the partitions of the function correctly, we can obtain the original expression with $f(x)\neq0 \forall x \in [a,b].F(x) := \int_{a}^{x} f(t) dt = 0.$
\end{solution}

\pagebreak
\begin{problem}[Exercise 2.16]
Assume $f$ and $g$ are differentiable functions on $[a,b]$ and assume $f', g' \in \mathcal{R}([a,b]).$ Show that the integration by parts formula is valid: $$\int_{a}^{b} fg'dx = f(b)g(b) - f(a)g(a) - \int_{a}^{b} f'gdx.$$ Make sure you show the relevant functions are Riemann integrable when you do the proof!
\end{problem}

\begin{solution}
Let's analyze the product of the functions, $fg$. Apply the product rule so $\frac{d}{dx}fg = f'g + fg'$. Apply the Fundamental Theorem of Calculus, $\int_{a}^{b} \frac{d}{dx}(fg)dx = \int_{a}^{b}fg'dx + \int_{a}^{b}fg'dx \rightarrow \int_{a}^{b}fg'dx = \int_{a}^{b}\frac{d}{dx}(fg) dx - \int_{a}^{b}fg'dx$. Evaluate the expression so $\int_{a}^{b}fg'dx = f(b)g(b) - f(a)g(a) - \int_{a}^{b}f'gdx$. By Theorem 2.10(b), $\frac{d}{dx}(fg) \in \mathcal{R}$ so both functions in integrals are Reimann integrable.
\end{solution}


\begin{problem}[Exercise 2.17]
Assume $g : [a, b] \rightarrow \R$ is differentiable, that $g'$ is continuous, and $M$ and $m$ are upper and lower bounds, respectively, for the function $g$. Assume $f : [m,M]\rightarrow\R$ is continuous. SHow that the change of variables formula is valid: $$\int_{a}^{b}f(g(x))g'(x)dx = \int_{g(a)}^{g(b)}f(t)dt.$$ Again, part of the exercise is to check that the relevant functions are Riemann integrable when you do the proof!
\end{problem}

\begin{solution}

\end{solution}

\begin{problem}[Exercise 2.18]
Assume $f \in \mathcal{R}([a,b]),$ but that $f$ has a jump discontinuity at $c \in (a,b)$, i.e. $f(c-)\neq f(c+).$ Show that $F(x) := \int_{a}^{x} f(t)dt$ is not differentiable at $x=c$.
\end{problem}

\begin{solution}

\end{solution}

\begin{problem}[Exercise 2.20]
Assume $g$ is bounded, $g \in \mathcal{R}([0,1])$ and $g$ is continuous at 0. Show that $$\lim_{n\rightarrow \infty}\int_{0}^{1}g(x^{n})dx = g(0).$$ Hint: Consider the difference $\int_{0}^{1}g(x^{n})dx - g(0)$; add and subtract $\int_{0}^{c} g(x^{n})dx$ for a carefully chosen $c$, and then that $\int_{0}^{c}dx$ is close to $cg(0)$ for large enough $n.$
\end{problem}

\begin{solution}

\end{solution}

\begin{problem}[Exercise 3.3]
Let $(f_n)$ be a sequence of real-valued, Riemann integrable functions on the interval $[a,b]$. Assume that $f_{n}(x)\rightarrow 0$ as $n\rightarrow\infty$ for each $x\in[a,b],$ and additionally, $$\sum_{n=1}^{\infty}|f_{n + 1}(x) - f_{n}(x)|$$ converges uniformly on $[a,b]$.
\begin{enumerate}
    \item Show that $\lim_{n \rightarrow\infty}\int_{a}^{b}f_n dx \rightarrow 0.$
    \item Is it necessarily the case that the series $\sum_{n=1}^{\infty}f_n$ converges uniformly? Give a proof or counterexample to support your answer.
\end{enumerate}
\end{problem}

\begin{solution}
\end{solution}



\end{document}