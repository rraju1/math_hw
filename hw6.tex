% HMC Math dept HW template example
% v0.04 by Eric J. Malm, 10 Mar 2005
\documentclass[12pt,letterpaper,boxed]{hmcpset}

% set 1-inch margins in the document
\usepackage[margin=1in]{geometry}

% include this if you want to import graphics files with /includegraphics
\usepackage{mathtools}
\usepackage{graphicx}
\DeclareMathOperator{\Lim}{Lim}
\DeclareMathOperator{\Int}{Int}
\DeclareMathOperator{\Img}{Im}
\DeclareMathOperator{\Cl}{Cl}
\DeclareMathOperator{\A}{\mathcal{A}}
\DeclareMathOperator{\N}{\mathbb{N}}
\DeclareMathOperator{\R}{\mathbb{R}}
\DeclareMathOperator{\Q}{\mathbb{Q}}
\DeclareMathOperator{\bigep}{\mathcal{E}}
\DeclarePairedDelimiter\abs{\lvert}{\rvert}%
\DeclarePairedDelimiter{\norm}{\lVert}{\rVert}


\newcommand*\xoverline[2][0.75]{%
    \sbox{\myboxA}{$\m@th#2$}%
    \setbox\myboxB\null% Phantom box
    \ht\myboxB=\ht\myboxA%
    \dp\myboxB=\dp\myboxA%
    \wd\myboxB=#1\wd\myboxA% Scale phantom
    \sbox\myboxB{$\m@th\overline{\copy\myboxB}$}%  Overlined phantom
    \setlength\mylenA{\the\wd\myboxA}%   calc width diff
    \addtolength\mylenA{-\the\wd\myboxB}%
    \ifdim\wd\myboxB<\wd\myboxA%
       \rlap{\hskip 0.5\mylenA\usebox\myboxB}{\usebox\myboxA}%
    \else
        \hskip -0.5\mylenA\rlap{\usebox\myboxA}{\hskip 0.5\mylenA\usebox\myboxB}%
    \fi}

% info for header block in upper right hand corner
\name{Ravi Raju}
\class{MA 521}
\assignment{Homework \#6}
\duedate{2/28/2018}

\begin{document}

\problemlist{2.24, 3.12, 3.26, 5.3, 5.7, 5.8, 5.12, 5.13.}

\begin{problem}[Exercise 2.24]
Let $(X, d)$ be a metric space. Show that if $X$ is totally bounded, then $X$ is bounded.
\end{problem}

\begin{solution}
If $X$ is totally bounded, then it can be covered by finitely many balls of radius $\epsilon$ if $\{x_1, x_2, \dots, x_n\} \in X$ s.t. $\bigcup_{i = 1}^{n} B_{(X, d)}(x_j, \epsilon)$. So, simply choose the $x_n$ which has the ball with the maximum radius, $r_m$ and construct another ball with this radius $ + r$ s.t. all balls are contained within $B_X(x, r_m + r)$. So, $X$ is bounded.
\end{solution}

\begin{problem}[Exercise 3.12]
Let $(X, d)$ be a metric space. Assume $F$ and $K$ are subsets of $X$, with $F$ closed and $K$ compact. Then $F\cap K$ is compact.
\end{problem}

\begin{solution}
If $K$ is a compact subset of $X$, then $K$ is closed and bounded in $X$. The intersection of closed sets is closed so $F\cap K$ is closed. By Thm 3.10, $F \cap K \subset K$ and $K$ is compact so $F \cap K$ is also compact.
\end{solution}

\begin{problem}[Exercise 3.26]
Give an example of a collection $\mathcal{A}$ of bounded subsets of $\R$ such that $\mathcal{A}$ has the finite intersection property, but $\bigcap_{A \in \mathcal{A}} A = \emptyset$. Hint: If $A \subset \R$ is bounded in $\R$, what else can prevent it from being compact?
\end{problem}

\begin{solution}

\end{solution}

\begin{problem}[Exercise 5.3]
Let $\A$ be a collection of convex subsets of $\R^{k}.$ Show that $B := \bigcap_{A \in \A} A$ is convex. 
\end{problem}

\begin{solution}
Let's do proof by contradiction. Let $B=\bigcap_{A \in \A} A$. Assume $B$ is not convex. Let $a, b \in B$ so then $\exists \, t\in [0, 1]$ s.t. $z \in (1 - t)a + tb \notin B$. But $z \in A \forall A \in \A \rightarrow z \notin B$ so $B \neq \bigcap_{A \in \A} A.$ This is clearly a contradiction so $B$ is convex.
\end{solution}

\begin{problem}[Exercise 5.7]
Let $(X, d)$ be a metric space and let $A$ and $B$ be disjoint subsets of $X$. Prove that if $A$ and $B$ are both open in $X$, then $A$ and $B$ are seperated.
\end{problem}

\begin{solution}
We need to show that $A \cap \overline{B} = B \cap \overline{A} = \emptyset.$ So, let's analyze the first statement: $\overline{A}\cap B = (A \cup \Lim_{X}(A))\cap B = (A \cap B)\cup(\Lim_{X}(A) \cap B).$ $A$ and $B$ are disjoint so the only set we need to be concerned with is $\Lim_{X}(A) \cap B$. Consider the intersection of $\Lim_X(A)\cap \Lim_X(B) = C.$ Without loss of generality, choose $x \in C \rightarrow x \in \Lim_X(A)$ and $\Lim_X(B)\not \subset B$ since $B$ is open. So, $x \notin \Lim_X(A)\cap B.$ So, $\overline{A}\cap B = \emptyset$. This holds true for the other case as well and so $A$ and $B$ are both seperated.  
\end{solution}

\begin{problem}[Exercise 5.8]
Let $E$ be a connected subset of a metric space $(X, d)$. Show that $\overline{E}$ is connected.
\end{problem}

\begin{solution}
If $E$ is connected, then $E \subset \Lim_X (E)$. If $E$ is connected, then $E$ has no isolated points. If $E$ had isolated points, then $\exists$ some $x\in E$ s.t. $x \notin \Lim_X (E)$. Thus, $\exists$ some neighbourhood $U$ of $x$ s.t. $U\cap\setminus E\{x\} = \emptyset.$ Then, $E$ can be written as the union of two seperated sets $E = E\setminus \{x\}\cup \{x\}$, implying $E$ is not connected which is false. Thus, $\overline{E}$ is connected. 
\end{solution}

\begin{problem}[Exercise 5.12]
Let $(X,d)$ be a metric space, and let $\mathcal{C}$ be a collection of connected subsets of $X$. Assume $A = \bigcap_{C \in \mathcal{C}} C$ is nonempty. Show that $B = \bigcup_{C \in \mathcal{C}} C$ is connected.
\end{problem}

\begin{solution}
 Let's solve this problem via proof by contrapositive. Let $B$ not be connected so this implies that $B=Z\cup Y$ s.t. $Z \cap \overline{Y}=Y \cap \overline{Z} = \emptyset.$ Take connected subset $C_1 \in \mathcal{C}$ s.t. $C_1 \subset B$. By Thm. 5.11, $C_1 \subset Z$ or $C_1 \subset Y.$ So $Z \cap Y = \emptyset \rightarrow \bigcap_{C \in \mathcal{C}} C \neq \emptyset.$ So $B$ is connected.
\end{solution}

\begin{problem}[Exercise 5.13]
Let $X = \R^{2}$. Give an example of a connected subset $E$ of $X$, such that $\Int_X(E)$ is not connected. Prove both that your set $E$ is connected and that its interior is not. ((Hint: Consider the
union of two convex sets joined at a point. You may assume without proof the fact that convexity implies
connectedness in $\R^{2}$.)
\end{problem}

\begin{solution}
Let $A$ and $B$ be convex sets in $\R^{2}$ s.t. $A$ is a closed ball with radius 1 centered at $(1, 0)$ and $B$ is a ball with radius 1 centered at $(-1, 0)$. Because we can assume that convexity implies connectedness, we can claim that both $A$ and $B$ are connected. If $\mathcal{C}$ is the collection of all connected subsets of $\R^{2}$ then by Exer. 5.12 and assuming that $\bigcap_{C \in \mathcal{C}} C \neq \emptyset$, $A \cup B$ is also connected. However, consider the point at (0, 0) and call it $x$. For any $\epsilon > 0$, $\exists y \in B_{\R^{2}}(0 , \epsilon)$ ((0, $-\epsilon$) for e.g.) s.t. $y \notin A, B.$ So, $x \notin \Int_{\R^{2}}(A \cup B)$. This leads to $\Int_{\R^{2}}((A \cup B)\setminus x)= \Int_{\R^{2}}((A\setminus x \cup B\setminus x)$ which reduces into two seperated sets expressed as a union. Thus, the interior is not connected.
\end{solution}


\end{document}
