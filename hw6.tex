% HMC Math dept HW template example
% v0.04 by Eric J. Malm, 10 Mar 2005
\documentclass[12pt,letterpaper,boxed]{hmcpset}

% set 1-inch margins in the document
\usepackage[margin=1in]{geometry}

% include this if you want to import graphics files with /includegraphics
\usepackage{mathtools}
\usepackage{graphicx}
\DeclareMathOperator{\Lim}{Lim}
\DeclareMathOperator{\Int}{Int}
\DeclareMathOperator{\Img}{Im}
\DeclareMathOperator{\Cl}{Cl}
\DeclareMathOperator{\A}{\mathcal{A}}
\DeclareMathOperator{\N}{\mathbb{N}}
\DeclareMathOperator{\R}{\mathbb{R}}
\DeclareMathOperator{\Q}{\mathbb{Q}}
\DeclareMathOperator{\bigep}{\mathcal{E}}
\DeclarePairedDelimiter\abs{\lvert}{\rvert}%
\DeclarePairedDelimiter{\norm}{\lVert}{\rVert}


\newcommand*\xoverline[2][0.75]{%
    \sbox{\myboxA}{$\m@th#2$}%
    \setbox\myboxB\null% Phantom box
    \ht\myboxB=\ht\myboxA%
    \dp\myboxB=\dp\myboxA%
    \wd\myboxB=#1\wd\myboxA% Scale phantom
    \sbox\myboxB{$\m@th\overline{\copy\myboxB}$}%  Overlined phantom
    \setlength\mylenA{\the\wd\myboxA}%   calc width diff
    \addtolength\mylenA{-\the\wd\myboxB}%
    \ifdim\wd\myboxB<\wd\myboxA%
       \rlap{\hskip 0.5\mylenA\usebox\myboxB}{\usebox\myboxA}%
    \else
        \hskip -0.5\mylenA\rlap{\usebox\myboxA}{\hskip 0.5\mylenA\usebox\myboxB}%
    \fi}

% info for header block in upper right hand corner
\name{Ravi Raju}
\class{MA 521}
\assignment{Homework \#6}
\duedate{2/28/2018}

\begin{document}

\problemlist{2.24, 3.12, 3.26, 5.3, 5.7, 5.8, 5.12, 5.13.}

\begin{problem}[Exercise 2.24]
Let $(X, d)$ be a metric space. Show that if $X$ is totally bounded, then $X$ is bounded.
\end{problem}

\begin{solution}
Let $X = \bigcup_{B_X} B_X(x , \epsilon) = B_X(x,r)$. Choose $\zeta > 0$ so $B_X(x,r)\subset B_X(x, r + \zeta)$ so $X$ is bounded. 
\end{solution}

\begin{problem}[Exercise 3.12]
Let $(X, d)$ be a metric space. Assume $F$ and $K$ are subsets of $X$, with $F$ closed and $K$ compact. Then $F\cap K$ is compact.
\end{problem}

\begin{solution}
If $K$ is a compact subset of $X$, then $K$ is closed and bounded in $X$. The intersection of closed sets is closed so $F\cap K$ is closed. By Thm 3.10, $F \cap K \subset K$ and $K$ is compact so $F \cap K$ is also compact.
\end{solution}

\begin{problem}[Exercise 3.26]
Give an example of a collection $\mathcal{A}$ of bounded subsets of $\R$ such that $\mathcal{A}$ has the finite intersection property, but $\bigcap_{A \in \mathcal{A}} A = \neq$. Hint: If $A \subset \R$ is bounded in $\R$, what else can prevent it from being compact?
\end{problem}

\begin{solution}

\end{solution}

\begin{problem}[Exercise 5.3]
Let $\A$ be a collection of convex subsets of $\R^{k}.$ Show that $B := \bigcap_{A \in \A} A$ is convex. 
\end{problem}

\begin{solution}
Let's do proof by contradiction. Let $B=\bigcap_{A \in \A} A$. Assume $B$ is not convex. Let $a, b \in B$ so then $\exists \, t\in [0, 1]$ s.t. $z \in (1 - t)a + tb \notin B$. But $z \in A \forall A \in \A \rightarrow z \notin B$ so $B \neq \bigcap_{A \in \A} A.$ This is clearly a contradiction so $B$ is convex.
\end{solution}

\begin{problem}[Exercise 5.7]
Let $(X, d)$ be a metric space and let $A$ and $B$ be disjoint subsets of $X$. Prove that if $A$ and $B$ are both open in $X$, then $A$ and $B$ are seperated.
\end{problem}

\begin{solution}
We need to show that $A \cap \overline{B} = B \cap \overline{A} = \emptyset.$ So, let's analyze the first statement: $\overline{A}\cap B = (A \cup \Lim_{X}(A))\cap B = (A \cap B)\cup(\Lim_{X}(A) \cap B).$ $A$ and $B$ are disjoint so the only set we need to be concerned with is $\Lim_{X}(A) \cap B$. Consider the intersection of $\Lim_X(A)\cap \Lim_X(B) = C.$ Without loss of generality, choose $x \in C \rightarrow x \in \Lim_X(A)$ and $\Lim_X(B)\not \subset B$ since $B$ is open. So, $x \notin \Lim_X(A)\cap B.$ So, $\overline{A}\cap B = \emptyset$. This holds true for the other case as well and so $A$ and $B$ are both seperated.  
\end{solution}


\end{document}
