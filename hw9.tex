% HMC Math dept HW template example
% v0.04 by Eric J. Malm, 10 Mar 2005
\documentclass[12pt,letterpaper,boxed]{hmcpset}

% set 1-inch margins in the document
\usepackage[margin=1in]{geometry}

% include this if you want to import graphics files with /includegraphics
\usepackage{mathtools}
\usepackage{graphicx}
\DeclareMathOperator{\Lim}{Lim}
\DeclareMathOperator{\Int}{Int}
\DeclareMathOperator{\Img}{Im}
\DeclareMathOperator{\Cl}{Cl}
\DeclareMathOperator{\A}{\mathcal{A}}
\DeclareMathOperator{\N}{\mathbb{N}}
\DeclareMathOperator{\R}{\mathbb{R}}
\DeclareMathOperator{\Q}{\mathbb{Q}}
\DeclareMathOperator{\bigep}{\mathcal{E}}
\DeclarePairedDelimiter\abs{\lvert}{\rvert}%
\DeclarePairedDelimiter{\norm}{\lVert}{\rVert}

\newcommand*\xoverline[2][0.75]{%
    \sbox{\myboxA}{$\m@th#2$}%
    \setbox\myboxB\null% Phantom box
    \ht\myboxB=\ht\myboxA%
    \dp\myboxB=\dp\myboxA%
    \wd\myboxB=#1\wd\myboxA% Scale phantom
    \sbox\myboxB{$\m@th\overline{\copy\myboxB}$}%  Overlined phantom
    \setlength\mylenA{\the\wd\myboxA}%   calc width diff
    \addtolength\mylenA{-\the\wd\myboxB}%
    \ifdim\wd\myboxB<\wd\myboxA%
       \rlap{\hskip 0.5\mylenA\usebox\myboxB}{\usebox\myboxA}%
    \else
        \hskip -0.5\mylenA\rlap{\usebox\myboxA}{\hskip 0.5\mylenA\usebox\myboxB}%
    \fi}

% info for header block in upper right hand corner
\name{Ravi Raju}
\class{MA 521}
\assignment{Homework \#9}
\duedate{4/18/2019}

\begin{document}

\problemlist{Chapter 7: 2.16, 2.17, 3.6, 3.7}

\begin{problem}[Exercise 2.16]
For each of the following sequences $(a_n)_{n = 1}^{\infty}$, prove whether the series $\sum_{n =1}^{\infty}a_n$ converges or diverges. (If it converges, you do not need to find the limit.)
\begin{enumerate}
    \item $a_n = \sqrt{n + 1} - \sqrt{n}$.
    \item $a_n = \frac{\sqrt{n + 1} - \sqrt{n}}{n}$.
    \item $a_n = (\sqrt[n]{n} - 1)^{n}$.
    \item $a_n = \frac{(-1)^{n}}{\log{n}}$ for $n \ge 2$ (and $a_1 = 0$).
\end{enumerate}
\end{problem}

\begin{solution}
\begin{enumerate}
    \item 
    \item 
    \item 
    \item 
\end{enumerate}
\end{solution}

\begin{problem}[Exercise 2.17]
Consider the series $$\sum_{n=1}^{\infty} \frac{1}{1 + z^{n}}.$$ Determine which values of $z\in\R (z \neq -1)$ make the series convergent and which make it divergent. Prove your answers are correct.
\end{problem}

\begin{solution}

\end{solution}


\begin{problem}[Exercise 3.6]
Assume that $\sum_{n = 1}^{\infty} a_n$ converges absolutely. Prove that $\sum_{n =1}^{\infty} \frac{\sqrt{\abs{a_n}}}{n}$ converges. (Hint: Use the inequality $2AB \leq A^{2} + B^{2},$ valid for any real numbers $A, B$).
\end{problem}


\begin{solution}

\end{solution}


\begin{problem}[Exercise 3.7]
\begin{enumerate}
    \item Assume that $\sum_{n = 1}^{\infty} a_n$ and $\sum_{n = 1}^{\infty} b_n$ converge absolutely. Prove that $\sum_{n = 1}^{\infty} (a_n + b_n)$ absolutely as well.
    \item Assume that $\sum_{n = 1}^{\infty} a_n$ converges. Does it follow that $\sum_{n = 1}^{\infty} a_2n$ converges? Give a proof or counterexample.
    \item Assume that $\sum_{n = 1}^{\infty} a_n$ converges absolutely. Does it follow that $\sum_{n = 1}^{\infty} a_2n$ converges absolutely? Give a proof or counterexample.
\end{enumerate}
\end{problem}

\begin{solution}
\begin{enumerate}
    \item 
    \item
    \item 
\end{enumerate}
\end{solution}

\end{document}