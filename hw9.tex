% HMC Math dept HW template example
% v0.04 by Eric J. Malm, 10 Mar 2005
\documentclass[12pt,letterpaper,boxed]{hmcpset}

% set 1-inch margins in the document
\usepackage[margin=1in]{geometry}

% include this if you want to import graphics files with /includegraphics
\usepackage{mathtools}
\usepackage{graphicx}
\DeclareMathOperator{\Lim}{Lim}
\DeclareMathOperator{\Int}{Int}
\DeclareMathOperator{\Img}{Im}
\DeclareMathOperator{\Cl}{Cl}
\DeclareMathOperator{\A}{\mathcal{A}}
\DeclareMathOperator{\N}{\mathbb{N}}
\DeclareMathOperator{\R}{\mathbb{R}}
\DeclareMathOperator{\Q}{\mathbb{Q}}
\DeclareMathOperator{\bigep}{\mathcal{E}}
\DeclarePairedDelimiter\abs{\lvert}{\rvert}%
\DeclarePairedDelimiter{\norm}{\lVert}{\rVert}

\newcommand*\xoverline[2][0.75]{%
    \sbox{\myboxA}{$\m@th#2$}%
    \setbox\myboxB\null% Phantom box
    \ht\myboxB=\ht\myboxA%
    \dp\myboxB=\dp\myboxA%
    \wd\myboxB=#1\wd\myboxA% Scale phantom
    \sbox\myboxB{$\m@th\overline{\copy\myboxB}$}%  Overlined phantom
    \setlength\mylenA{\the\wd\myboxA}%   calc width diff
    \addtolength\mylenA{-\the\wd\myboxB}%
    \ifdim\wd\myboxB<\wd\myboxA%
       \rlap{\hskip 0.5\mylenA\usebox\myboxB}{\usebox\myboxA}%
    \else
        \hskip -0.5\mylenA\rlap{\usebox\myboxA}{\hskip 0.5\mylenA\usebox\myboxB}%
    \fi}

% info for header block in upper right hand corner
\name{Ravi Raju}
\class{MA 521}
\assignment{Homework \#9}
\duedate{4/18/2019}

\begin{document}

\problemlist{Chapter 7: 2.16, 2.17, 3.6, 3.7}

\begin{problem}[Exercise 2.16]
For each of the following sequences $(a_n)_{n = 1}^{\infty}$, prove whether the series $\sum_{n =1}^{\infty}a_n$ converges or diverges. (If it converges, you do not need to find the limit.)
\begin{enumerate}
    \item $a_n = \sqrt{n + 1} - \sqrt{n}$.
    \item $a_n = \frac{\sqrt{n + 1} - \sqrt{n}}{n}$.
    \item $a_n = (\sqrt[n]{n} - 1)^{n}$.
    \item $a_n = \frac{(-1)^{n}}{\log{n}}$ for $n \ge 2$ (and $a_1 = 0$).
\end{enumerate}
\end{problem}

\begin{solution}
\begin{enumerate}
    \item Enumerate partial sums of $a_n$. $s_n=(\sqrt{2} - \sqrt{1}) + (\sqrt{3} - \sqrt{2}) \dots + (\sqrt{k + 1} - \sqrt{k}) = -1 + (\sqrt{2} - \sqrt{2}) \dots + (\sqrt{k} - \sqrt{k}) + \sqrt{k + 1}.$ So, $\sqrt{k + 1} - 1$ as $n\rightarrow\infty$ diverges so series diverges.
    \item Multiply numerator and denominator by $\frac{\sqrt{n + 1} + \sqrt{n}}{\sqrt{n + 1} + \sqrt{n}}$. So, $\frac{\sqrt{n + 1} - \sqrt{n}}{n} = \frac{1}{n\sqrt{n + 1} + \sqrt{n}} < n^{\frac{-3}{2}}$. According to Theorem 2.4, $\frac{\sqrt{n + 1} - \sqrt{n}}{n}$ diverges.
    \item Use Root Test so $\lim_{n \rightarrow\infty} \sup(\sqrt[n]{n} - 1) < 1. \lim_{n \rightarrow\infty} \sup \sqrt[n]{n} < 2. \lim_{n \rightarrow\infty} \sup \frac{\log{n}}{n} < \log{2}$. $\lim_{n \rightarrow\infty} \sup n < 2^{n} \rightarrow \frac{n}{2^{n}} < 1$, which is true for all  n. So series converges. 
    \item Use Alternating Series Test to show that $\frac{1}{\log{n}}$ is monotonically decreasing. For all $n \ge 2, \frac{1}{\log{n + 1}} < \frac{1}{\log{n}}$. So by AST, this series converges.
\end{enumerate}
\end{solution}

\begin{problem}[Exercise 2.17]
Consider the series $$\sum_{n=1}^{\infty} \frac{1}{1 + z^{n}}.$$ Determine which values of $z\in\R (z \neq -1)$ make the series convergent and which make it divergent. Prove your answers are correct.
\end{problem}

\begin{solution}
Let's use the Ratio Test. $\abs{\frac{a_{n+1}}{a_n}} = \abs{(\frac{1+z}{1 + z^{2}}, \frac{1+z^{2}}{1+z^{3}},\frac{1+z^{3}}{1+z^{4}},\dots, \frac{1+z^{n}}{1+z^{n+1}})}$. So, $\lim_{n\rightarrow \infty} \sup \abs{\frac{1+z^{n}}{1+z^{n+1}}} = \abs{\frac{z^{n}}{z^{n+1}}} = \frac{1}{\abs{z}} < 1$. If $\abs{z}>1$, then $\sum_{n}^{\infty}\frac{1}{1 + z^{n}}$ converges. Now, $\lim_{n\rightarrow \infty} \inf \abs{\frac{1+z^{n}}{1+z^{n+1}}} = \abs{\frac{z^{n}}{z^{n+1}}} = \frac{1}{\abs{z}} > 1$. If $\abs{z} < 1, \sum_{n=1}^{\infty} \frac{1}{1 + z^{n}}$ diverges. If $z=1$, $\sum_{n=1}^{\infty} \frac{1}{1 + 1^{n}}$. We know that $\sum_{n=1}^{\infty} \frac{1}{2}$ diverges so by comparison test, $\sum_{n=1}^{\infty} \frac{1}{1 + z^{n}}$ diverges for $z=1$.
\end{solution}


\begin{problem}[Exercise 3.6]
Assume that $\sum_{n = 1}^{\infty} a_n$ converges absolutely. Prove that $\sum_{n =1}^{\infty} \frac{\sqrt{\abs{a_n}}}{n}$ converges. (Hint: Use the inequality $2AB \leq A^{2} + B^{2},$ valid for any real numbers $A, B$).
\end{problem}


\begin{solution}
If $\sum_{n = 1}^{\infty} a_n$ converges absolutely, then $\sum_{n = 1}^{\infty} \abs{a_n}$ converges. Consider any term in $\sum_{n =1}^{\infty} \frac{\sqrt{\abs{a_n}}}{n}$ and set it equal to $AB$ such that $A = \sqrt{\abs{a_n}}, B = \frac{1}{n}$. So, $2\frac{\sqrt{\abs{a_n}}}{n} \leq \abs{a_n} + \frac{1}{n^{2}} \forall\,n\in\N$. By comparison test, $2\frac{\sqrt{\abs{a_n}}}{n}$ converges since $\abs{a_n} + \frac{1}{n^{2}}$ converges. So, $\frac{\sqrt{\abs{a_n}}}{n}$ also must converge.
\end{solution}


\begin{problem}[Exercise 3.7]
\begin{enumerate}
    \item Assume that $\sum_{n = 1}^{\infty} a_n$ and $\sum_{n = 1}^{\infty} b_n$ converge absolutely. Prove that $\sum_{n = 1}^{\infty} (a_n + b_n)$ absolutely as well.
    \item Assume that $\sum_{n = 1}^{\infty} a_n$ converges. Does it follow that $\sum_{n = 1}^{\infty} a_{2n}$ converges? Give a proof or counterexample.
    \item Assume that $\sum_{n = 1}^{\infty} a_n$ converges absolutely. Does it follow that $\sum_{n = 1}^{\infty} a_{2n}$ converges absolutely? Give a proof or counterexample.
\end{enumerate}
\end{problem}

\begin{solution}
\begin{enumerate}
    \item If $\sum_{n = 1}^{\infty} a_n$ and $\sum_{n = 1}^{\infty} b_n$ absolutely converge, then $\sum_{n = 1}^{\infty} \abs{a_n}$ and $\sum_{n = 1}^{\infty} \abs{b_n}$ converge. By triangle inequality, $\sum_{n = 1}^{\infty} \abs{a_n + b_n} \leq \sum_{n = 1}^{\infty} \abs{a_n} + \sum_{n = 1}^{\infty} \abs{b_n}$. By comparison test, $\sum_{n = 1}^{\infty} \abs{a_n + b_n}$ converges. So, $\sum_{n = 1}^{\infty} (a_n + b_n)$ absolutely converges.
    \item No. Consider the series $\sum_{n=1}^{\infty} \frac{(-1)^{n}}{n}$, which converges by AST. But, $\sum_{n=1}^{\infty} \frac{(-1)^{2n}}{2n}= \sum_{n=1}^{\infty} \frac{(1)}{2n}$ is a divergent series.
    \item Yes. If $\sum_{n = 1}^{\infty} a_n$ converges absolutely, then $\sum_{n = 1}^{\infty} \abs{a_n}$ converges. For any $k\in\N, \sum_{n = 1}^{k} \abs{a_{2n}} \leq \sum_{n = 1}^{k} \abs{a_n} \leq \sum_{n = 1}^{\infty} \abs{a_n}.$ So, $\sum_{n = 1}^{\infty} abs{a_{2n}}$ converges which implies $\sum_{n = 1}^{\infty} a_{2n}$ converges absolutely.
\end{enumerate}
\end{solution}

\end{document}